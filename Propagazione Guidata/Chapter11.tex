\chapter{Parametri ABCD di un Doppio Bipolo}
Per l'analisi di \textbf{bipoli} in \textbf{cascata} è conveniente che la \textbf{tensione} e la \textbf{corrente} in \textbf{uscita dal primo doppio bipolo} risultino \textbf{l'ingresso al secondo}:
\begin{squared}[violet]
\begin{dcases}
    V_1 = A V_2 + B I_2\\
    I_1 = C V_2 + D I_2
\end{dcases}    
\end{squared}
Dove:
\begin{itemize}
    \item $A = \frac{V_1}{V_2}\big|_{I_2 = 0}$ \textbf{Adimensionale} Inverso di un \textbf{Guadagno} di \textbf{tensione} a \textbf{vuoto}
    \item $B = \frac{V_1}{I_2}\big|_{V_2 = 0}$ \textbf{Transimpedenza} tra ingresso e uscita \textbf{cortocircuitata}
    \item $C = \frac{I_1}{V_2}\big|_{I_2 = 0}$ \textbf{Transimpedenza} tra ingresso e uscita \textbf{aperta}
    \item $D = \frac{I_1}{I_2}\big|_{V_2 = 0}$ \textbf{Adimensionale} Inverso di un \textbf{Guadagno} di \textbf{corrente} di \textbf{cortocircuito}
\end{itemize}
Un \textbf{doppio bipolo lineare} può essere descritto anche dai \textbf{parametri Z}:
\begin{equation*}
    \begin{dcases}
    V'_1 = Z_{11} I'_1 + Z_{12} I'_2\\
    V'_2 = Z_{21} I'_1 + Z_{22} I'_2
    \end{dcases}
\end{equation*}
Manipoliamo il sistema \textbf{ABCD} per ricondurci al \textbf{sistema dei parametri Z}:
\begin{equation*}
    V_2 = \frac{1}{C} I_1 - \frac{D}{C} I_2
\end{equation*}
Che sostituiamo nella prima:
\begin{equation*}
\begin{aligned}
    V_1 &= \frac{A}{C} I_1 - \frac{AD}{C} I_2 + B I_2 =\\
    &= \frac{A}{C} I_1 - \frac{(AD - BC)}{C} I_2
\end{aligned}    
\end{equation*}

Quindi abbiamo ottenuto:
\begin{equation*}
    \begin{dcases}
        V_1 = \frac{A}{C} I_1 - \frac{(AD - BC)}{C} I_2\\
        V_2 = \frac{1}{C} I_1 - \frac{D}{C} I_2
    \end{dcases}
\end{equation*}
Notando le seguenti uguaglianze tra i due sistemi:
\begin{equation*}
    \begin{dcases}
        V_1 = V'_1\\
        I_1 = I'_1\\
        V_2 = V'_2\\
        I_2 = - I'_2
    \end{dcases}
\end{equation*}
Che sostituiamo nel sistema appena ottenuto:
\begin{equation*}
    \begin{dcases}
        V'_1 = \frac{A}{C} I'_1 + \frac{(AD - BC)}{C} I'_2\\
        V'_2 = \frac{1}{C} I'_1 + \frac{D}{C} I'_2
    \end{dcases}
\end{equation*}
Quindi identifichiamo:
\begin{equation*}
    \begin{bmatrix}
        Z_{11} = \frac{A}{C} & Z_{12} = \frac{AD-BC}{C}\\
        Z_{21} = \frac{1}{C} & Z_{22} = \frac{D}{C}
    \end{bmatrix}
\end{equation*}
Quindi possiamo \textbf{sfruttare le proprietà dimostrate nel capitolo dei parametri Z}, come la \textbf{reciprocità}:
\begin{equation*}
    Z_{12} = Z_{21} \implies \frac{1}{C} = \frac{AD-BC}{C} \implies AD - BC = 1
\end{equation*}
E \textbf{l'assenza di perdite}:
\begin{equation*}
    Re\left\{Z_{11}\right\} = Re\left\{\frac{A}{C}\right\} = 0
\end{equation*}
\begin{equation*}
    Re\left\{Z_{22}\right\} = Re\left\{\frac{D}{C}\right\} = 0
\end{equation*}
\begin{equation*}
    Re\left\{Z_{12}\right\} =  Re\left\{Z_{21}\right\} = Re\left\{\frac{1}{C}\right\} = 0
\end{equation*}
Che si verificano se:
\begin{itemize}
    \item $Re\left\{C\right\} = 0$
    \item $Im\left\{A\right\} = 0$
    \item $Im\left\{D\right\} = 0$
    \item $Re\left\{B\right\} = 0$
\end{itemize}