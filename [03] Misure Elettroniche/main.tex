\documentclass[titlepage, oneside]{book}
\usepackage{fancyhdr}
\usepackage[italian]{babel}
\pagestyle{fancy}
\fancyheadoffset[RO,LE]{0.4in}
\fancyhead[L]{\leftmark }
\fancyhead[R]{\thepage}
\fancyfoot[C]{\thepage}
\setlength{\headsep}{0.3in}
\fancyfoot[R]{\href{https://linktr.ee/achillecannavale}{Achille Cannavale}}
\usepackage{pdfpages}
\usepackage[utf8]{inputenc}
\usepackage[a4paper, total={6in, 10in}]{geometry}
\renewcommand{\thesection}{\arabic{section}}
\usepackage{algorithm}
\usepackage{algpseudocode}
\usepackage{dsfont}

%Per le immagini%
\usepackage{graphicx} 
\usepackage{xcolor}

%Per la Matematica%
\usepackage{mathrsfs,amsmath,amssymb}
\usepackage[makeroom]{cancel}
\usepackage{etoolbox}


%Font Family%
\usepackage{mathptmx}

\usepackage{esvect}

%Element Font Size%
\usepackage{sectsty}
\sectionfont{\fontsize{18}{20}\selectfont}
\subsectionfont{\fontsize{16}{16}\selectfont}
\subsubsectionfont{\fontsize{12}{12}\selectfont}

\usepackage{nccmath}
\usepackage{pgfplots}

\usepackage{caption}
\usepackage{subcaption}


%Colori%
\usepackage{empheq}
\usepackage[most]{tcolorbox}
\usepackage{framed}

%Per i link nell'indice%
\usepackage{hyperref}
\hypersetup{
    colorlinks,
    citecolor=black,
    filecolor=black,
    linkcolor=black,
    urlcolor=black
}
\usepackage{circuitikz}
\usepackage{wrapfig}
%SHORTCUT-----------------------------------------------------------
\newcommand{\rot}[1]{\bar{\nabla} \times  #1}
\newcommand{\dive}[1]{\bar{\nabla} \cdot  #1}
\newcommand{\parti}[2]{\frac{\partial #1}{\partial #2}}
\newcommand{\und}[1]{\underline{#1}}
\newcommand{\w}{\omega}
\newcommand{\h}{\underline{h}}
\newcommand{\e}{\underline{e}}
\renewcommand{\H}{\underline{H}}
\newcommand{\E}{\underline{E}}
\renewcommand{\b}{\underline{b}}
\renewcommand{\d}{\underline{d}}
\renewcommand{\j}{\underline{j}}
\newcommand{\J}{\underline{J}}
\renewcommand{\k}{\underline{k}}
\renewcommand{\r}{\underline{r}}

%SHORTCUT-----------------------------------------------------------

\AtBeginEnvironment{equation*}{\LARGE}
\usepackage{environ}
\usepackage{fontawesome5}
\usepackage{amsfonts} 

  
  \usepackage{colonequals}
\newcommand*{\logeq}{\ratio\Leftrightarrow}
\usetikzlibrary{circuits.ee.IEC}
\newcommand\esymbol[1]{\tikz[circuit ee IEC] \draw (0,0) -- (.1,0) node [#1,anchor=west,name=s] {} (s.east) -- +(.1,0);}

%-------------------------------------------------------%
\newtcbox{\mymath}[1][]{%
    nobeforeafter, math upper, tcbox raise base,
    enhanced, colframe=blue!30!black,
    colback=blue!30, boxrule=1pt,
    #1}
\NewEnviron{squared}[1][cyan]{
\begin{empheq}[box={\mymath[colback=#1!10,drop lifted shadow, rounded corners]}]{equation*}
\LARGE
    \BODY
\end{empheq}
}

\begin{document}


\begin{titlepage}
\centering

 \vspace*{\fill}
    {\Huge\bfseries
Misure Elettroniche\par
}
\vspace{6ex}
{\Large
Riassunto da
\par}

        \vspace{1.5cm}
            
        \textbf{\large Achille Cannavale}
            
        \vfill

        \vspace{0.8cm}

   
\end{titlepage}

\large
\tableofcontents
\Large
\chapter*{Introduzione}
Lo scopo di questa tesi è la progettazione e l'implementazione di un game engine in C++ specializzato nella verifica e nella risoluzione delle collisioni tra corpi rigidi rettangolari. 

L'obiettivo è quello di creare un'architettura solida e performante che possa essere utilizzata per lo sviluppo di giochi 2D di differenti generi, come platformer, metroidvania, arcade e gestionali.
\chapter{Metodo Voltamperometrico}
Questo metodo serve per misurare \textbf{resistenze elettriche}:
\begin{center}
    \includegraphics[width=.4\textwidth]{Images/figure2.png}
\end{center}
Per determinare $R_x$ viene alimentata in \textbf{continua} e vengono misurate \textbf{tensione} e \textbf{corrente}.\\ \\
Sono possibili due tipi di collegamenti:
\begin{itemize}
    \item \textbf{Metodo voltamperometrico a monte dell'amperometro}
    \item \textbf{Metodo voltamperometrico a valle dell'amperometro}
\end{itemize}
\section{Metodo voltamperometrico a monte dell'amperometro}
Il \textbf{voltmetro a monte} è preferibile quando si devono misurare \textbf{resistenze elevate}, ossia confrontabili con la \textbf{resistenza interna al voltmetro} $R_v$:
\begin{center}
    \includegraphics[width=.4\textwidth]{Images/figure3.png}
\end{center}
\begin{equation*}
\begin{dcases}
     I_x = I_m\\
    V_m = V_A + V_x
\end{dcases}
\end{equation*}
Quindi:
\begin{equation*}
    R_x = \frac{V_x}{I_x} = \frac{V_m - V_A}{I_m} = \underbrace{\frac{V_m}{I_m}}_{R_m} - R_A
\end{equation*}
Di solito $R_A$ (chiamata anche \textbf{errore di consumo}) ci viene data dal costruttore.\\ \\
\subsection{Incertezza}
\begin{equation*}
    u_{R_x} = \sqrt{u^2_{\left(\frac{V_m}{I_m}\right)} + u^2_{R_A}} 
\end{equation*}
dove:
\begin{equation*}
    u^2_{\left(\frac{V_m}{I_m}\right)} ={\left(\frac{V_m}{I_m}\right)}^2\left(\dot{u}^2_{V_m} + \dot{u}^2_{I_m} \right)
\end{equation*}
\begin{equation*}
    u_{R_A} = \frac{\Delta R_A}{\sqrt{3}}
\end{equation*}

\section{Metodo voltamperometrico a valle dell'amperometro}
Questa configurazione si utilizza per misurare \textbf{resistenze piccole} in rapporto alla \textbf{resistenza interna del voltmetro}:
\begin{center}
    \includegraphics[width=.4\textwidth]{Images/figure4.png}
\end{center}
\begin{equation*}
    \begin{dcases}
        V_m = V_x\\
        I_m = I_x + I_v
    \end{dcases}
\end{equation*}
Quindi (lavoreremo più facilmente con le \textbf{ammettenze}):
\begin{equation*}
    G_x = \frac{I_m - I_v}{V_m} = G_m - G_v
\end{equation*}
\subsection{Incertezza}
\begin{equation*}
    u_{G_x} = \sqrt{u^2_{\left(\frac{I_m}{V_m}\right)} + u^2_{G_v}}
\end{equation*}
dove:
\begin{equation*}
   u^2_{\left(\frac{I_m}{V_m}\right)}=  \left(\frac{I_m}{V_m}\right)^2 \cdot \left(\dot{u}^2_{I_m} + \dot{u}^2_{V_m} \right)
\end{equation*}
\chapter{Potenza lungo una Linea senza Perdite}
\section{Linea indefinita}
\begin{center}
    \includegraphics[width=0.8\textwidth]{Images/Figure13.png}
\end{center}
Lungo questa \textbf{linea senza perdite} si propaga la sola onda:
\begin{equation*}
    \begin{dcases}
    V(z) = V^+ e^{-jkz}\\
    I(z) = \frac{V^+}{Z_0} e^{-jkz}
    \end{dcases}
\end{equation*}
Valutiamo ora la potenza su un'ascissa generica:
\begin{equation*}
    P(z) = \frac{1}{2} V(z) I^*(z) = \frac{1}{2} V^+ e^{-jkz} \frac{{V^+}^*}{Z_0}e^{jkz} =  \frac{1}{2} \frac{|V^+|^2}{Z_0}
\end{equation*}
\section{Due Linee Indefinite}
Consideriamo ora il seguente schema:
\begin{center}
    \includegraphics[width=0.8\textwidth]{Images/Figure14.png}
\end{center}
\begin{equation*}
\tag{Prima Linea}
\begin{dcases}
    V_0(z) = V_0^+ e^{-jk_0z} +  V_0^- e^{jk_0z} = V_0^+ e^{-jk_0z} + V_0^+ \Gamma(0) e^{jk_0z} \\
    I_0(z) = \frac{V_0^+}{Z_0} e^{-jk_0z} - \frac{V_0^-}{Z_0} e^{jk_0z} = \frac{V_0^+}{Z_0} e^{-jk_0z} - \frac{V_0^+}{Z_0}  \Gamma(0) e^{jk_0z}
\end{dcases}
\end{equation*}
Dove:
\begin{equation*}
    \Gamma(0) = \frac{V_0^-}{V_0^+} = \frac{Z_1 - Z_0}{Z_1 + Z_0} 
    \begin{dcases}
    Positivo \ se \ Z_1 > Z_0 \\
    Negativo \ se \ Z_1 < Z_0
    \end{dcases}
\end{equation*}
Mentre nella \textbf{seconda linea}:
\begin{equation*}
\tag{Seconda Linea}
\begin{dcases}
    V_1(z) = V_1^+ e^{-jk_1z}\\
    I_1(z) = \frac{V_1^+}{Z_1} e^{-jk_1z} 
\end{dcases}
\end{equation*}

Per la \textbf{Continuità della Tensione} possiamo dire che:
\begin{equation*}
    V_0(0) = V_1(0)
\end{equation*}
Quindi:
\begin{equation*}
    V_0^+ + V_0^- = V_1^+
\end{equation*}
Possiamo quindi riscrivere $V_1^+$ come:
\begin{equation*}
    V_1^+ = V_0^+ (1 + \Gamma(0)) =  V_0^+ \left(1 + \frac{Z_1 - Z_0}{Z_1 + Z_0} \right) =V_0^+   \left(\frac{2 \ Z_1 }{Z_1 + Z_0} \right)
\end{equation*}
La \textbf{Potenza della seconda linea} è:
\begin{equation*}
\begin{aligned}
       P_1(z) &= \frac{1}{2} V_1(z) I_1^*(z) =  \frac{1}{2} V_1^+ e^{-jk_1z} \frac{V_1^*}{Z_1} e^{jk_1z} =\\
       &=\frac{1}{2} \frac{|V_1^+|^2}{Z_1} = \frac{1}{2} {\left(\frac{2 \ Z_1 }{Z_1 + Z_0} \right)}^2 \frac{|V_0^+|^2}{Z_1} = \\
       &= \frac{2 \ Z_1 }{{(Z_1 + Z_0)}^2}|V_0^+|^2
\end{aligned}
\end{equation*}
Mentre la \textbf{Potenza nella prima linea} è:
\begin{equation*}
\begin{aligned}
       P_0(z) &= \frac{1}{2} V_0(z) I_0^*(z) =  \frac{1}{2}\left(V_0^+ e^{-jk_0z} + V_0^+ \Gamma(0) e^{jk_0z}\right)\\ &{\left(\frac{V_0^+}{Z_0} e^{-jk_0z} - \frac{V_0^+}{Z_0} \Gamma(0) e^{jk_0z}\right)}^* =\\
       &=\frac{1}{2}\left(V_0^+ e^{-jk_0z} + V_0^+ \Gamma(0) e^{jk_0z}\right) \left(\frac{{V_0^+}^*}{Z_0} e^{jk_0z} - \frac{{V_0^+}^*}{Z_0} \Gamma(0) e^{-jk_0z}\right) = \\
       &= \frac{1}{2} \frac{|V_0^+|^2}{Z_0} - {\Gamma(0)}^2 \frac{1}{2} \frac{|V_0^+|^2}{Z_0} + \frac{1}{2} \frac{|V_0^+|^2}{Z_0} \Gamma(0) e^{2jk_0z} - \\
       &- \frac{1}{2} \frac{|V_0^+|^2}{Z_0} \Gamma(0) e^{-2jk_0z}
\end{aligned}
\end{equation*}
Quindi:\footnote{$\frac{1}{2} e^a - \frac{1}{2} e^{-a} = j \sin(a)$}
\begin{equation*}
    P_0(z) = \frac{1}{2} \frac{|V_0^+|^2}{Z_0} - {\Gamma(0)}^2 \frac{1}{2} \frac{|V_0^+|^2}{Z_0} + j \frac{|V_0^+|^2}{Z_0} \Gamma(0) \sin(2jk_0z)
\end{equation*}
Quindi la Potenza nella sezione generica z è costituita da una \textbf{Potenza Attiva} indipendente da z:
\begin{equation*}
    \tag{Potenza Attiva}
    Re\{P_0(z)\} = \frac{1}{2} \frac{|V_0^+|^2}{Z_0} - {\Gamma(0)}^2 \frac{1}{2} \frac{|V_0^+|^2}{Z_0}
\end{equation*}
E da una \textbf{Potenza Reattiva}:
\begin{equation*}
    \tag{Potenza Reattiva}
    Im\{P_0(z)\} = \frac{|V_0^+|^2}{Z_0} \Gamma(0) \sin(2jk_0z)
\end{equation*}
Questa potenza si \textbf{annulla} per $z=0$, cioè se si calcola proprio sul \textbf{carico}, essendo resistivo, all'aumentare di z aumenta anche il \textbf{modulo della potenza reattiva}, raggiungendo il suo \textbf{massimo} in $z = - \frac{\lambda}{8}$ per poi diminuire ed \textbf{annullarsi} in $z= - \frac{\lambda}{4}$.\\
(ripetendosi con periodo $\frac{\lambda}{2}$)
\newpage
\section{Linea con Carico}
Prendiamo ora il seguente caso:
\begin{center}
    \includegraphics[width=0.8\textwidth]{Images/figure15.png}
\end{center}
Dove avremo i seguenti andamenti di tensione e corrente lungo la linea:
\begin{equation*}
    \begin{dcases}
    V(z) = V^+ e^{-jkz}  V^- e^{jkz} = V^+ e^{-jkz} + V^+ \Gamma(0) e^{jkz} = \\
    \quad= V^+ e^{-jkz} + V^+ |\Gamma(0)| e^{j(kz + \Phi_0)}\\
    I(z) = \frac{V^+}{Z_0} + e^{-jkz} - \frac{V^-}{Z_0} e^{jkz} = \frac{V^+}{Z_0} e^{-jkz} - \frac{V^+}{Z_0} |\Gamma(0)| e^{j(kz + \Phi_0)} 
    \end{dcases}
\end{equation*}

E in generale la potenza vale:
\begin{equation*}
    \begin{aligned}
    P(z) &= \frac{1}{2} V(z) I^*(z) = \frac{1}{2} \left( V^+ e^{-jkz} + V^+ |\Gamma(0)| e^{j(kz + \Phi_0)}\right) \\
    &{\left(\frac{V^+}{Z_0} e^{-jkz} - \frac{V^+}{Z_0} |\Gamma(0)| e^{j(kz + \Phi_0)} \right)}^* =\\
    =& \frac{1}{2} \left( V^+ e^{-jkz} + V^+ |\Gamma(0)| e^{j(kz + \Phi_0)}\right) \\
    &\left(\frac{{V^+}^*}{Z_0} e^{jkz} - \frac{{V^+}^*}{Z_0} |\Gamma(0)| e^{-j(kz + \Phi_0)} \right)=\\
    &= \frac{1}{2} \frac{|V^+|^2}{Z_0} - |\Gamma(0)|^2 \frac{1}{2} \frac{|V^+|^2}{Z_0} + \frac{1}{2} \frac{|V^+|^2}{Z_0}  |\Gamma(0)| e^{j(kz+\Phi_0)} - \\
       &- \frac{1}{2} \frac{|V^+|^2}{Z_0} |\Gamma(0)| e^{-j(kz+\Phi_0)}
    \end{aligned}
\end{equation*}
Quindi:
\begin{equation*}
    P(z) = \frac{1}{2} \frac{|V^+|^2}{Z_0} - |\Gamma(0)|^2 \frac{1}{2} \frac{|V^+|^2}{Z_0} + j \frac{|V^+|^2}{Z_0} |\Gamma(0)| \sin(2kz +\Phi_0)
\end{equation*}
Anche in questo caso la potenza alla sezione z è costituita da una \textbf{Potenza Attiva} indipendente da z:
\begin{equation*}
    \tag{Potenza Attiva}
    Re\{P(z)\} = \frac{1}{2} \frac{|V^+|^2}{Z_0} - |\Gamma(0)|^2 \frac{1}{2} \frac{|V^+|^2}{Z_0}
\end{equation*}
E da una \textbf{Potenza Reattiva}:
\begin{equation*}
    \tag{Potenza Reattiva}
    Im\{P(z)\} = \frac{|V^+|^2}{Z_0} |\Gamma(0)| \sin(2kz + \Phi_0)
\end{equation*}
Che si \textbf{annulla} per $2kz + \Phi_0 = n\pi$, quindi \textbf{non si annulla} per $z=0$ come nel caso di un carico \textbf{puramente resistivo}.\\ \\
Infatti nel caso di un carico con una \textbf{parte reattiva non nulla} ($Z_l = R_l + j X_l$), è associata la \textbf{potenza reattiva}:
\begin{equation*}
    Im\{P(0)\} = \frac{1}{2} |I(0)|^2 X_l = \frac{|V^+|^2}{Z_0} |\Gamma(0)| \sin(\Phi_0) = 2\w (w_m^l - w_e^l)
\end{equation*}
Dove $w_m^l \ e \ w_e^l$ sommo le \textbf{energie pseudo magnetiche e pseudo elettriche immagazzinate nel carico}.


























\chapter{Potenza lungo la Linea con Perdite}
Consideriamo il\textbf{ circuito equivalente} associato ad 
una \textbf{fettina dz} di \textbf{linea con perdite}:
\begin{center}
    \includegraphics[width=0.8\textwidth]{Images/figure16.png}
\end{center}
Che possiamo sintetizzare nel seguente modo:
\begin{center}
    \includegraphics[width=0.8\textwidth]{Images/figure17.png}
\end{center}
Dove:
\begin{itemize}
    \item $Z_{RL} = j\w L_{eq}= j\w \left(L -     j\frac{R}{\w}  \right) = j\w L + R$
    \item $Y_{GC} = j\w C_{eq}= j\w \left(C -     j\frac{G}{\w}  \right) = j\w C + G$
\end{itemize}
\textbf{Risolvendo} il circuito ottengo:
\begin{equation*}
\tag{Tensione}
    V(z + dz) - V(z) = -Z_{RL} dz I = - j \w L_{eq} dz I
\end{equation*}
Quindi:
\begin{equation*}
    \frac{dV}{dz} = - j \w L_{eq} I
\end{equation*}
\begin{equation*}
\tag{Corrente}
    I(z + dz) - I(z) = -Y_{GC} dz V = - j \w C_{eq} dz V
\end{equation*}
Quindi:
\begin{equation*}
    \frac{dI}{dz} = - j \w C_{eq} V
\end{equation*}
\section{Linea Indefinita}
In particolare valutiamo la seguente situazione:
\begin{center}
    \includegraphics[width=0.8\textwidth]{Images/figure18.png}
\end{center}
Dove avremo il seguente andamento di \textbf{tensione} e \textbf{corrente}:
\begin{equation*}
    \begin{dcases}
    V(z) = V^+ e^{-jkz} + V^- e^{jkz}\\
    I(z) = \frac{V^+}{Z_0} e^{-jkz} + \frac{V^-}{Z_0} e^{jkz}
    \end{dcases}
\end{equation*}
Dove:
\begin{equation*}
\begin{aligned}
    k &= \w \sqrt{L_{eq} C_{eq}} = \w \sqrt{\left(L - j\frac{R}{\w} \right) \left(C - j\frac{G}{\w}  \right)} =  \\
    &=\w \sqrt{\left(LC - \frac{RG}{\w^2} \right)- j\left(\frac{LG}{\w} + \frac{RC}{\w} \right)}
\end{aligned}
\footnote{fai la moltiplicazione e poi separa Re e Im}
\end{equation*}
Quindi:
\begin{equation*}
    k = \beta - j \alpha
\end{equation*}
E consideriamo il solo \textbf{caso fisico}:
\begin{equation*}
    \beta >0 \ , \quad \alpha >0
\end{equation*}
\begin{equation*}
    V(z) = V^+ e^{-jkz} = V^+ e^{-j(\beta - j\alpha )z} = |V^+| e^{j\alpha_+} e^{-\alpha z} e^{-j\beta z} 
\end{equation*}
E passiamo nel \textbf{dominio del tempo}:
\begin{equation*}
    v(z,t) = |V^+| e^{-\alpha z} \cos(\w t + \alpha_+ - \beta z)
\end{equation*}
Che con $\beta>0$ e $\alpha>0$ rappresenta \textbf{un'onda} che si propaga con \textbf{velocità} $v = \frac{\w}{\beta}$ nelle z positive \textbf{attenuandosi}.\\ \\
Complessivamente la \textbf{tensione} sarà:
\begin{equation*}
    V(z) = V^+ e^{-jkz} + V^- e^{jkz} = V^+ e^{-\alpha z} e^{-j\beta z} + V^- e^{-\alpha z} e^{j\beta z}
\end{equation*}
Calcoliamoci $Z_0$:
\begin{equation*}
    Z_0 = \sqrt{\frac{L_{eq}}{C_{eq}}} = \sqrt{\frac{L - j \frac{R}{\w}}{C - j \frac{G}{\w}}} = \sqrt{\frac{L}{C} \ \frac{1 - j \frac{R}{\w L} }{1 - j \frac{G}{\w C}}}
\end{equation*}
$Z_0$ sarà sempre \textbf{reale} e \textbf{positivo} se vale:
\begin{equation*}
\tag{HEAVISIDE}
    \frac{R}{\w L} = \frac{G}{\w C} 
\end{equation*}
In tal caso:
\begin{equation*}
    Z_0 = \sqrt{\frac{L}{C}} 
\end{equation*}
Se L, C, R e G \textbf{non dipendono dalla frequenza} allora:
\begin{itemize}
    \item $\beta = \w \sqrt{LC}$
    \item $\alpha = - \sqrt{\frac{L}{C}} G$
\end{itemize}
Tuttavia la condizione di \textbf{HEAVISIDE} è difficile da verificarsi, quindi nel caso in cui non si verifichi, la scelta \textbf{corretta} è quella di $Re\{Z_0\} >0$:
\begin{equation*}
    \begin{dcases}
    V(z) = V^+ e^{-j(\beta - j\alpha)z}\\
    I(z) = \frac{V^+}{Z_0} e^{-j(\beta - j\alpha)z}
    \end{dcases}
\end{equation*}
Calcolo la \textbf{Potenza}:
\begin{equation*}
    P(z) = \frac{1}{2} V(z) I^*(z) = \frac{1}{2} \frac{|V^+|^2}{Z_0^*} e^{-2\alpha z}
\end{equation*}
Suddivido in \textbf{potenza attiva} e \textbf{potenza reattiva}:
\begin{equation*}
    P(z) = \frac{1}{2} \frac{|V^+|^2}{\left(\frac{|Z_0|^2}{Z_0}\right)} e^{-2\alpha z} = \frac{1}{2} \frac{|V^+|^2}{R_0^2 +X_0^2} (R_0 + j X_0) e^{-2\alpha z}
\end{equation*}
\begin{equation*}
    P_R(z) =  \frac{1}{2} \frac{|V^+|^2}{R_0^2 +X_0^2}R_0 e^{-2\alpha z}
\end{equation*}
Che è una \textbf{potenza positiva} e \textbf{diminuisce} al \textbf{crescere} \textbf{di} \textbf{z}.\\ \\
\section{Potenza tra due ascisse}
Per capire meglio analizziamo il seguente caso:
\begin{center}
    \includegraphics[width=0.8\textwidth]{Images/figure19.png}
\end{center}
Dove avremo le seguenti \textbf{potenze}:
\begin{equation*}
    \begin{dcases}
    P_R(z_1) = \frac{1}{2}\frac{|V^+|^2}{|Z_0|^2} R_0 e^{-2\alpha z_1}\\
    P_R(z_2) = \frac{1}{2}\frac{|V^+|^2}{|Z_0|^2} R_0 e^{-2\alpha z_2}
    \end{dcases}
\end{equation*}
\begin{equation*}
\begin{aligned}
    P_R(z_1) - P_R(z_2) &= \frac{1}{2}\frac{|V^+|^2}{|Z_0|^2} R_0 e^{-2\alpha z_1} - \frac{1}{2}\frac{|V^+|^2}{|Z_0|^2} R_0 e^{-2\alpha z_2} = \\
    &= \frac{1}{2}\frac{|V^+|^2}{|Z_0|^2} R_0\left(e^{2\alpha l_1} - e^{2\alpha l_2}\right)
\end{aligned}
\end{equation*}
La \textbf{differenza delle due potenze attive} rappresenta la \textbf{potenza dissipata dalla linea tra le due sezioni} ed è positiva (quindi $R_0 >0$, $\alpha>0$).\\ \\
Ripetiamo il calcolo per $P_x$:
\begin{equation*}
    \begin{dcases}
    P_X(z_1) = \frac{1}{2}\frac{|V^+|^2}{|Z_0|^2} X_0 e^{-2\alpha z_1}\\
    P_X(z_2) = \frac{1}{2}\frac{|V^+|^2}{|Z_0|^2} X_0 e^{-2\alpha z_2}
    \end{dcases}
\end{equation*}
\begin{equation*}
\begin{aligned}
    P_X(z_1) - P_X(z_2) &= \frac{1}{2}\frac{|V^+|^2}{|Z_0|^2} X_0 e^{-2\alpha z_1} - \frac{1}{2}\frac{|V^+|^2}{|Z_0|^2} X_0 e^{-2\alpha z_2} = \\
    &= \frac{1}{2}\frac{|V^+|^2}{|Z_0|^2} X_0\left(e^{2\alpha l_1} - e^{2\alpha l_2}\right)
\end{aligned}
\end{equation*}
La \textbf{differenza tra le potenze reattive} è \textbf{proporzionale alla differenza delle pseudo energie elettriche e magnetiche medie immagazzinate nel tratto di linea}.\\ \\
Tale \textbf{potenza} è \textbf{positiva} o \textbf{negativa} in base al segno di $X_0$ (discorso analogo per $V^-$).\\ \\
Consideriamo ora:
\begin{equation*}
\begin{aligned}
    k &= \w \sqrt{\left(LC - \frac{RG}{\w^2} \right)- j\left(\frac{LG}{\w} + \frac{RC}{\w} \right)} =\\
    &=\w \sqrt{LC}\sqrt{1 - \frac{RG}{\w^2 LC} - j\left(\frac{G}{\w C} + \frac{R}{\w L} \right) }
\end{aligned}
\end{equation*}
Avere \textbf{piccole perdite} vuol dire:
\begin{itemize}
    \item $\frac{R}{\w L} << 1 \implies \frac{RG}{\w^2 LC} << \frac{R}{\w L}$\\
     \item $\frac{G}{\w C} << 1 \implies \frac{RG}{\w^2 LC} << \frac{G}{\w C}$
\end{itemize}
Queste approssimazioni ci \textbf{semplificano} l'espressione di k:
\begin{equation*}
    k \approx \w \sqrt{LC} \left(1 - j \frac{1}{2} \left(\frac{G}{\w C} + \frac{R}{\w L}\right)\right) = \beta  - j \alpha
\end{equation*}
Quindi $\beta$ ha lo steso valore di \textbf{k} in \textbf{assenza di perdite}, ma esse danno luogo ad un'\textbf{attenuazione} che cresce \textbf{linearmente} con le \textbf{perdite}.\\ \\
Calcoliamo ora l'\textbf{impedenza caratteristica}:
\begin{equation*}
\begin{aligned}
    Z_0 &= \sqrt{\frac{L_{eq}}{C_{eq}}} = \sqrt{\frac{L - j \frac{R}{\w}}{C - j \frac{G}{\w}}} = \sqrt{\frac{L}{C} \frac{1 - j \frac{R}{\w L}}{1 - j \frac{G}{\w C}}} =\\
    &= \sqrt{\frac{L}{C}} {\left(1 - j \frac{R}{\w L}\right)}^{1/2}{\left(1 - j \frac{G}{\w C}\right)}^{-1/2}
\end{aligned}
\end{equation*}
Nel caso di \textbf{piccole perdite}:
\begin{equation*}
    \frac{R}{\w L} << 1 \implies {\left(1 - j \frac{R}{\w L}\right)}^{1/2} \approx 1 - j \frac{1}{2} \frac{R}{\w L}
\end{equation*}
E dato che:
\begin{equation*}
    \frac{G}{\w C} << 1 \implies {\left(1 - j \frac{G}{\w C}\right)}^{1/2} \approx 1 + j \frac{1}{2} \frac{G}{\w C}
\end{equation*}
Quindi sostituendo:\footnote{Trascuro il termine con il coefficiente $\frac{1}{4}$}
\begin{equation*}
    Z_0 = \sqrt{\frac{L}{C}} + j \frac{1}{2} \sqrt{\frac{L}{C}} \left(\frac{G}{\w C} - \frac{R}{\w L} \right) = R_0 + j X_0
\end{equation*} 
Dove:
\begin{itemize}
    \item $R_0 = \sqrt{\frac{L}{C}}$
    \item $X_0 = \frac{1}{2} \sqrt{\frac{L}{C}} \left(\frac{G}{\w C} - \frac{R}{\w L} \right)$
\end{itemize}
Quindi ha la \textbf{parte reale} uguale a come sarebbe $Z_0$ in \textbf{mancanza di perdite}, o nel caso in cui valga la \textbf{condizione di Heaviside}.\\ \\
\section{Linea con Carico}
Infine consideriamo la seguente situazione:
\begin{center}
    \includegraphics[width=0.5\textwidth]{Images/figure20.png}
\end{center}
Dove:
\begin{equation*}
    \begin{dcases}
    k = \beta - j \alpha\\
    Z_0 = R_0 + j X_0
    \end{dcases}
\end{equation*}
\begin{equation*}
\begin{aligned}
    V(z) &= V^+ e^{-\alpha z} e^{-j\beta z} + V^- e^{\alpha z} e^{j\beta z} = \\
    &= V^+ e^{-\alpha z} e^{-j\beta z} \left(1 + \frac{V^-}{V^+} e^{2\alpha z} e^{2j\beta z}\right) = V^+ e^{-\alpha z} e^{-j\beta z} \left(1 + \Gamma(z)\right)\\
    I(z) &= \frac{V^+}{Z_0} e^{-\alpha z} e^{-j\beta z} \left(1 - \Gamma(z)\right)
\end{aligned}
\end{equation*}
\begin{equation*}
    \Gamma(0) = \Gamma_l = \frac{V^-}{V^+} = \frac{Z_l - Z_0}{Z_l + Z_0}
\end{equation*}
Nel caso di piccole perdite $X_0 << R_0$, quindi:
\begin{equation*}
    \Gamma(0) = \frac{Z_l - R_0 - j X_0}{Z_l + R_0 + j X_0} \approx \frac{Z_l - R_0}{Z_l + R_0} - j \underbrace{\frac{X_0}{R_0} \frac{1}{\frac{Z_l}{R_0} + 1}}_{trascurare}
\end{equation*}
La differenza maggiore dal caso \textbf{senza perdite} è che $|\Gamma(z)| =|\Gamma(0)| e^{2\alpha z} $ \textbf{non è costante} lungo la linea, ma allontanandosi dal \textbf{carico}, il modulo \textbf{diminuisce}.\\ \\
Calcoliamo ora la generica potenza lungo la linea:\footnote{$|\Gamma(z)| = |\Gamma_L| e^{2\alpha z}$}\footnote{$\Gamma(z) = \Gamma_L e^{2\alpha z} e^{2j\beta z}$}
\begin{equation*}
    \begin{aligned}
    P(z) &= \frac{1}{2} V(z) I^*(z) = \frac{1}{2} V^+ e^{-\alpha z} e^{-j \beta z} \left(1 + \Gamma(z) \right) \frac{{V^+}^*}{Z_0^*}e^{-\alpha z} e^{j \beta z} \left(1 - {\Gamma(z)}^* \right) =\\
    &= \frac{1}{2} \frac{|V^+|^2}{Z_0^*} e^{-2\alpha z}\left(1 - |\Gamma(z)|^2 + \Gamma(z) - {\Gamma(z)}^* \right) =\\
    &= \frac{1}{2} \frac{|V^+|^2}{Z_0^*} e^{-2\alpha z}\left(1 - |\Gamma(z)|^2 +2j Im\{\Gamma(z)\} \right) =\\
    &= \frac{1}{2} \frac{|V^+|^2}{Z_0^*} e^{-2\alpha z}\left(1 - |\Gamma_l|^2 e^{4\alpha z}\right) + j  \frac{|V^+|^2}{Z_0^*} e^{-2\alpha z}e^{2\alpha z} Im\{\Gamma_l e^{2j\beta z}\}=\\
    &= \underbrace{\frac{1}{2} \frac{|V^+|^2}{Z_0^*} e^{-2\alpha z}\left(1 - {|\Gamma_l|}^2 e^{4\alpha z}\right)}_{\text{P. onda dir.\ meno P. onda rifl.}} + j  \frac{{|V^+|}^2}{Z_0^*}Im\{\Gamma_l e^{2j\beta z}\}
    \end{aligned}
\end{equation*}

Allontanandosi dal carico la \textbf{potenza} \textbf{dell'onda riflessa} sarà sempre più \textbf{piccola} di quella dell'\textbf{onda incidente}.







\chapter{Adattamenti}
\section{Adattamento con Stub Serie}
\subsection{Metodo 1}
Poniamoci nella situazione in cui vogliamo collegare il carico $Z_l$ sulla linea \textbf{senza} \textbf{perdite}, con \textbf{impedenza} \textbf{caratteristica} $Z_0$:
\begin{center}
    \includegraphics[width=0.8\textwidth]{Images/figure21.png}
\end{center}
Se la colleghiamo direttamente, avremo un \textbf{coefficiente di riflessione} non nullo:
\begin{equation*}
    \Gamma_l = \frac{Z_l - Z_0}{Z_l + Z_0}
\end{equation*}
E un \textbf{ROS} pari a:
\begin{equation*}
    ROS = \frac{1 + |\Gamma_l|}{1 - |\Gamma_l|}
\end{equation*}
Spostandosi lungo la linea, $|\Gamma|$ rimane \textbf{costante}, mentre il \textbf{modulo dell'impedenza} cambia sezione per sezione:
\begin{equation*}
    \frac{Z_0}{ROS} \leq |Z(z)| \leq Z_0 \ ROS
\end{equation*}
E analogamente la \textbf{parte reale dell'impedenza} varia tra:
\begin{equation*}
    \frac{Z_0}{ROS} \leq R(z) \leq Z_0 \ ROS   
\end{equation*}
In particolare esiste una coordinata $z'$ tale che:
\begin{equation*}
    R(z') = Z_0
\end{equation*}
E allo stesso tempo:
\begin{equation*}
    Z(z') = Z_0 + j \underbrace{X(z')}_{X_s}
\end{equation*}
\begin{center}
    \includegraphics[width=0.7\textwidth]{Images/figure22.png}
\end{center}
\textbf{Ma come posso individuare questa sezione $z'$?\\ \\
}Ricordiamo che:
\begin{equation*}
\begin{aligned}
     Z(z') &= Z_0 \ \frac{1 + \Gamma(z')}{1 - \Gamma(z')} = Z_0 \ \frac{[1 + \Gamma(z')][1 - {\Gamma(z')}^*]}{{|1 - \Gamma(z')|}^2} =\\
     &= Z_0 \ \frac{1 - |\Gamma|^2 + 2j Im\{\Gamma(z')\}}{|1 - \Gamma(z')|^2} =\\
     &=Z_0 \ \frac{1 - |\Gamma|^2 }{|1 - \Gamma(z')|^2} + j Z_0 2 \frac{Im\{\Gamma(z')\}}{|1 - \Gamma(z')|^2}
\end{aligned}
\end{equation*}
Quindi affinchè:
\begin{equation*}
    R(z') = Z_0 \leftrightarrow \frac{1 - |\Gamma|^2 }{|1 - \Gamma(z')|^2} = 1
\end{equation*}
(se $|\Gamma|=1$ la linea è chiusa su una pura reattanza)\\ \\
Se $|\Gamma|<1$:
\begin{equation*}
    \begin{aligned}
    &1 - {|\Gamma|}^2 = {|1 - \Gamma(z')|}^2 = [1 - \Gamma(z')][1 - {\Gamma(z')}^*] = \\
    &= 1 - {\Gamma(z')}^* - \Gamma(z') + {|\Gamma|}^2 = 1 - 2Re\{\Gamma(z')\} + {|\Gamma|}^2
    \end{aligned}
\end{equation*}
Quindi:
\begin{equation*}
    Re\{\Gamma(z')\} = |\Gamma|^2 = |\Gamma_l|^2
\end{equation*}
Ricordando che:
\begin{equation*}
    \begin{aligned}
    \Gamma(z) &= \Gamma(0) e^{2jkz} = \Gamma_l e^{2jkz} = |\Gamma_l| e^{j\varphi_l} e^{2jkz} =|\Gamma_l| e^{j(2kz +\varphi_l)} =\\
    &= |\Gamma_l| \cos(2kz +\varphi_l) + j |\Gamma_l| \sin(2kz +\varphi_l)
    \end{aligned}
\end{equation*}
Quindi sostituendo otterremo:
\begin{equation*}
    Re\{\Gamma(z')\} = |\Gamma_l| \cos(2kz' +\varphi_l) = |\Gamma|^2 \implies \cos(2kz' +\varphi_l) = |\Gamma|
\end{equation*}
Che ha come soluzioni:
\begin{equation*}
    2k z'_n = \frac{4\pi}{\lambda} \pm ar\cos(|\Gamma_n|) - \varphi_l + 2n\pi
\end{equation*}
\begin{equation*}
    \implies z'_n = \pm \frac{ar\cos(|\Gamma_n|)}{2\pi} \frac{\lambda}{2} - \frac{\varphi_l}{2\pi} \frac{\lambda}{2} + n \frac{\lambda}{2} \ \forall n \ : \ z'_n <0
\end{equation*}
Quindi abbiamo trovato le ascisse $z'_n$ in corrispondenza delle quali:
\begin{equation*}
    Z(z'_n) = Z_0 + j Z_0 2 \frac{Im\{\Gamma(z'_n)\}}{|1 - \Gamma(z'_n)|^2}
\end{equation*}
Ricordando anche che:
\begin{itemize}
    \item $1 - |\Gamma_l|^2 = |1 - \Gamma(z_n')|^2 $
    \item $Im\{\Gamma(z'_n)\} = |\Gamma_l| \sin(2kz'_n +\varphi_l) = \pm \sqrt{1 - |\Gamma_l|^2}$
\end{itemize}

Quindi:
\begin{equation*}
    X_s = Im\{Z(z'_n)\} = Z_0 2 \frac{Im\{\Gamma(z'_n)\}}{|1 - \Gamma(z'_n)|^2} =  \pm Z_0 \underbrace{\frac{\sqrt{1 - |\Gamma_l|^2}}{1 - |\Gamma_l|^2}}_{\frac{\sqrt{X}}{X} = \frac{1}{\sqrt{X}}} = \frac{\pm Z_0}{\sqrt{1 - |\Gamma_l|^2}}
\end{equation*}
Quindi per ottenere l'adattamento dobbiamo \textbf{compensare} la $X_s$ inserendo a $z'_n$ una $X_s$ \textbf{uguale} ed \textbf{opposta}:
\begin{center}
    \includegraphics[width=0.7\textwidth]{Images/figure23.png}
\end{center}
La \textbf{reattanza} $X_s$ può essere realizzata con uno \textbf{stub in corto} o con \textbf{uno stub aperto} (es.\ corto):
\begin{center}
    \includegraphics[width=0.7\textwidth]{Images/figure24.png}
\end{center}
\textbf{Ma a che distanza devo mettere lo stub??}\\ \\
\begin{equation*}
    Z(l) = j Z_0 \tan(kl)
\end{equation*}
Quindi pongo:\footnote{La tangente si annulla ogni $n\pi$}
\begin{equation*}
   Z_0 \tan(kl_{stub}) = - X_s
\end{equation*}
\begin{equation*}
  \implies l_{stub} = - atan\left(\frac{X_s}{Z_0}\right) \frac{\lambda}{2\pi} + n \frac{\lambda}{2} \geq 0
\end{equation*}
\subsection{Metodo 2}
Gli stessi risultati li possiamo ottenere con il \textbf{trasporto di impedenza}:\\ \\
Voglio trovare tutte le sezioni in cui \textbf{l'impedenza di carico trasportata ha parte reale $Z_0$}.
\begin{equation*}
    Z(l) = Z_0 \frac{R_l + j Z_0 \tan(kl)}{Z_0 + j R_l \tan(kl)} = Z_0 + j X_s
\end{equation*}
Divido primo e secondo membro per $Z_0$ e numeratore e denominatore per $Z_0$:
\begin{equation*}
    Z'(l) = \frac{R'_l + j t}{1 + j R'_l t} = 1 + j X'_s
\end{equation*}
Dove:
\begin{itemize}
    \item $Z'(l) = \frac{Z(l)}{Z_0}$
    \item $R'(l) = \frac{R_l}{Z_0}$
    \item $X'_s = \frac{X_s}{Z_0}$
    \item $t = \tan(kl)$
\end{itemize}
In particolare:
\begin{equation*}
    \begin{dcases}
    Re\{Z'(l)\} = 1\\
    Im\{Z'(l)\} = X'_s
    \end{dcases}
\end{equation*}
Cerchiamo ora di separare \textbf{parte} \textbf{reale} e \textbf{parte} \textbf{immaginaria}:
\begin{equation*}
    Z'(l) = \frac{(R'_l + j t)(1 - j R'_l t)}{1 +  {R'_l}^2 t^2} = \frac{R'_l - j {R'_l}^2 + jt + R'_l t^2 }{1 + {R'_l}^2 t^2} = 1 + j X'_s
\end{equation*}
Quindi:
\begin{equation*}
    \frac{R'_l + R'_l t^2}{1 +  {R'_l}^2 t^2} = 1
\end{equation*}
\begin{equation*}
    \frac{- {R'_l}^2 t +t}{1 + {R'_l}^2 t^2} = X'_s
\end{equation*}
\textbf{Risolviamo la prima in t}:\\ 
Ricordiamoci però che la vera incognita è l!!\\ \\
Quindi per prima cosa verifichiamo se $t=\infty$, ovvero $l = \frac{\lambda}{4}$ è \textbf{soluzione del problema}:
\begin{equation*}
    t=\infty \implies Z\left(\frac{\lambda}{4}\right) \approx \frac{1}{R'_l} = 1
\end{equation*}
Ma questo può accadere solo se $R'_l = 1$, quindi\textbf{ NON è soluzione.}\\ \\
Procediamo quindi moltiplicando per $1 + {R'_l}^2 t^2$:
\begin{equation*}
    R'_l + {R'_l} t^2= 1 + {R'_l}^2 t^2 \implies ( {R'_l}^2 - R'_l) t^2 = R'_l -1
\end{equation*}
\begin{equation*}
    \implies t^2 = \frac{R'_l -1}{{R'_l}^2 - R'_l} \implies t^2 = \frac{\cancel{R'_l -1}}{R'_l\cancel{({R'_l} - 1)}} = \frac{1}{R'_l} \implies t = \frac{1}{\sqrt{R'_l}}
\end{equation*}
\begin{equation*}
    \implies \tan(kl) = t = \pm \sqrt{\frac{1}{{R'_l}}} = \pm \sqrt{\frac{Z_0}{R_l}}
\end{equation*}
Da questa espressione valuto la l in corrispondenza della quale:
\begin{equation*}
    Z(l) = Z_0 + j X_s
\end{equation*}
Ovvero:
\begin{equation*}
    l = \\arctan\left(\sqrt{\frac{Z_0}{R_l}}\right) \frac{\lambda}{2\pi} + n \frac{\lambda}{2} \geq 0
\end{equation*}
Una volta calcolata la l alla quale vogliamo fare l'adattamento mi ricavo $X_s$:
\begin{equation*}
    \begin{aligned}
    X'_s &= \frac{- {R'_l}^2 t +t}{1 + {R'_l}^2 t^2} = \frac{\pm (1 - {R'_l}^2) \sqrt{\frac{1}{R'_l}}}{1 + R'_l} = \pm \frac{\frac{1 - {R'_l}^2}{\sqrt{R'_l}}}{1 + R'_l} =\\
    &= \pm \frac{1 -{R'_l}^2 }{\sqrt{R'_l} (1 +{R'_l})} = \pm \frac{(1 -{R'_l})\cancel{(1 +{R'_l})}}{\sqrt{R'_l} \cancel{(1 +{R'_l})}} = \pm \frac{1 - {R'_l}}{\sqrt{R'_l}}
    \end{aligned}
\end{equation*}
Quindi:
\begin{equation*}
    X_s = X'_s \cdot \underbrace{Z_0}_{R_0} = \pm \frac{1 - \frac{R_l}{Z_0}}{\sqrt{\frac{R_l}{Z_0}}} \cdot Z_0 = \pm \frac{Z_0 - R_l}{\sqrt{R_l} / \sqrt{Z_0}} = \pm (R_0 - R_l) \cdot \sqrt{\frac{R_0}{R_l}}
\end{equation*}
Questa \textbf{reattanza} va compensata con una \textbf{reattanza} \textbf{uguale} e \textbf{opposta} attraverso uno \textbf{stub serie}:
\begin{equation*}
    X_{stub} = Z_0 \tan(kl_{stub}) = - X_s
\end{equation*}
\begin{equation*}
    kl_{stub} = -\arctan\left(\frac{X_s}{Z_0}\right) + n\pi
\end{equation*}
\begin{equation*}
   \implies l_{stub} = -\arctan\left(\frac{X_s}{Z_0}\right) \frac{\lambda}{2\pi} + n \frac{\lambda}{2} \geq 0
\end{equation*}

\section{Adattamento con Stub Parallelo}
\subsection{Metodo 1}
Nel caso di un \textbf{adattamento} mediante \textbf{stub parallelo} conviene lavorare con le \textbf{ammettenze}.\\
Consideriamo quindi una linea \textbf{senza perdite} con un'\textbf{ammettenza caratteristica} $Y_0 = \frac{1}{Z_0}$ che vogliamo collegare ad un carico di \textbf{ammettenza} $Y_l = \frac{1}{Z_l}$:
\begin{center}
    \includegraphics[width=0.7\textwidth]{Images/figure25.png}
\end{center}
Ricordiamo che:
\begin{equation*}
\begin{aligned}
     Y(z') &= Y_0 \ \frac{1 - \Gamma(z')}{1 + \Gamma(z')} = Y_0 \ \frac{[1 - \Gamma(z')][1 + {\Gamma(z')}^*]}{{|1 + \Gamma(z')|}^2} =\\
     &= Y_0 \ \frac{1 - |\Gamma|^2 - 2j Im\{\Gamma(z')\}}{|1 + \Gamma(z')|^2} =\\
     &=Y_0 \ \frac{1 - |\Gamma|^2 }{|1 + \Gamma(z')|^2} - j Y_0 2 \frac{Im\{\Gamma(z')\}}{{|1 + \Gamma(z')|}^2}
\end{aligned}
\end{equation*}
Quindi affinchè:
\begin{equation*}
    Y(z') = G(z') + j B(z') = Y_0 + j B_p \leftrightarrow Y_0 \frac{1 - |\Gamma|^2 }{|1 + \Gamma(z')|^2} = Y_0
\end{equation*}
(se $|\Gamma|=1$ la linea è chiusa su una pura reattanza)
Se $|\Gamma|<1$:
\begin{equation*}
    \begin{aligned}
    &1 - {|\Gamma|}^2 = {|1 + \Gamma(z')|}^2 = [1 + \Gamma(z')][1 + {\Gamma(z')}^*] = \\
    &= 1+ {\Gamma(z')}^* + \Gamma(z') + |\Gamma|^2 = 1 + 2Re\{\Gamma(z')\} + |\Gamma|^2
    \end{aligned}
\end{equation*}
Quindi:
\begin{equation*}
    Re\{\Gamma(z')\} = -|\Gamma|^2 = -|\Gamma_l|^2
\end{equation*}
Ricordando che:
\begin{equation*}
    \begin{aligned}
    \Gamma(z) &= \Gamma(0) e^{2jkz} = \Gamma_l e^{2jkz} = |\Gamma_l| e^{j\varphi_l} e^{2jkz} =|\Gamma_l| e^{j(2kz +\varphi_l)} =\\
    &= |\Gamma_l| \cos(2kz +\varphi_l) + j |\Gamma_l| \sin(2kz +\varphi_l)
    \end{aligned}
\end{equation*}
Quindi sostituendo otterremo:
\begin{equation*}
    Re\{\Gamma(z')\} = |\Gamma_l| \cos(2kz' +\varphi_l) = -|\Gamma|^2 \implies \cos(2kz' +\varphi_l) = -|\Gamma|
\end{equation*}
Che ha come \textbf{soluzioni}:
\begin{equation*}
    2k z'_n = \frac{4\pi}{\lambda} \pm ar\cos(-|\Gamma_n|) - \varphi_l + 2n\pi
\end{equation*}
\begin{equation*}
    \implies z'_n = \pm \frac{ar\cos(-|\Gamma_n|)}{2\pi} \frac{\lambda}{2} - \frac{\varphi_l}{2\pi} \frac{\lambda}{2} + n \frac{\lambda}{2} \ \forall n \ : \ z'_n <0
\end{equation*}
Che sono le \textbf{ascisse} in cui $G = Y_0$.\\ \\
Ripetendo calcoli analoghi:
\begin{equation*}
    Im\{\Gamma(z'_n)\} = |\Gamma_l| \sin(2kz'_n +\varphi_l) = \pm \sqrt{1 - |\Gamma_l|^2}
\end{equation*}
Quindi:
\begin{equation*}
    B_p = Im\{Y(z'_n)\} = Y_0 2 \frac{Im\{\Gamma(z'_n)\}}{|1 + \Gamma(z'_n)|^2} = \pm Y_0 \frac{\sqrt{1 - |\Gamma_l|^2}}{1 - |\Gamma_l|^2} = \frac{\pm Y_0}{\sqrt{1 - |\Gamma_l|^2}}
\end{equation*}
Quindi ora per ottenere l'\textbf{adattamento} devo compensare la \textbf{suscettanza} $B_p$ con una \textbf{suscettanza} \textbf{uguale} ed \textbf{opposta}, attraverso per esempio uno \textbf{stub parallelo} in corto:
\begin{center}
    \includegraphics[width=0.7\textwidth]{Images/figure26.png}
\end{center}
\subsection{Metodo 2}
Gli stessi risultati li possiamo ottenere con il \textbf{trasporto di ammettenza}:\\ \\
Voglio trovare tutte le sezioni in cui l'\textbf{ammettenza} di carico \textbf{trasportata} ha \textbf{parte reale} $G_L$.
\begin{equation*}
    Y(l) = Y_0 \frac{Y_l + j Y_0 \tan(kl)}{Y_0 + j Y_l \tan(kl)} = Y_0 + j B_p
\end{equation*}
Divido primo e secondo membro per $Y_0$ e numeratore e denominatore per $Y_0$:
\begin{equation*}
    Y'(l) = \frac{G'_l + j t}{1 + j G'_l t} = 1 + j B_p
\end{equation*}
Dove:
\begin{itemize}
    \item $Y'(l) = \frac{Y(l)}{Y_0}$
    \item $G'_l = \frac{G_l}{Y_0}$
    \item $B'_p = \frac{B_p}{Y_0}$
    \item $t = \tan(kl)$
\end{itemize}
In particolare:
\begin{equation*}
    \begin{dcases}
    Re\{Y'(l)\} = 1\\
    Im\{Y'(l)\} = B'_P
    \end{dcases}
\end{equation*}
Cerchiamo ora di separare \textbf{parte} \textbf{reale} e \textbf{parte} \textbf{immaginaria}:
\begin{equation*}
    Y'(l) = \frac{(G'_l + j t)(1 - j G'_l t)}{1 + j {G'_l}^2 t^2} = \frac{G'_l - j {G'_l}^2 + jt + G'_l t^2 }{1 + j {G'_l}^2 t^2} = 1 + j B'_P
\end{equation*}
Quindi:
\begin{equation*}
    \frac{G'_l + G'_l t^2}{1 + j {G'_l}^2 t^2} = 1
\end{equation*}
\begin{equation*}
    \frac{- {G'_l}^2 t +t}{1 + {G'_l}^2 t^2}
\end{equation*}
Risolviamo la prima in t:\\ 
Procediamo quindi moltiplicando per $1 + {G'_l}^2 t^2$:
\begin{equation*}
    G'_l + {G'_l}^2 = 1 + {G'_l}^2 t^2 \implies ( {G'_l}^2 - G'_l) t^2 = G'_l -1
\end{equation*}
\begin{equation*}
    \implies t^2 = \frac{G'_l -1}{{G'_l}^2 - G'_l} \implies t^2 = \frac{G'_l -1}{G'_l\cancel{({G'_l} - 1)}} = \frac{1}{G'_l} \implies t = \frac{1}{\sqrt{G'_l}}
\end{equation*}
\begin{equation*}
    \implies \tan(kl) = t = \pm \sqrt{\frac{1}{{G'_l}}} = \pm \sqrt{\frac{Y_0}{G_l}}
\end{equation*}
Da questa espressione valuto la l in corrispondenza della quale:
\begin{equation*}
    Y(l) = Y_0 + j B_p
\end{equation*}
Ovvero:
\begin{equation*}
    l = \arctan\left(\frac{Y_0}{G_l}\right) \frac{\lambda}{2\pi} + n \frac{\lambda}{2} \geq 0
\end{equation*}
Una volta calcolata la l alla quale vogliamo fare l'\textbf{adattamento} mi ricavo $B_p$:
\begin{equation*}
    \begin{aligned}
    B'_p &= \frac{- {G'_l}^2 t +t}{1 + {G'_l}^2 t^2} = \frac{\pm (1 - {G'_l}^2) \sqrt{\frac{1}{G'_l}}}{1 + G'_l} = \pm \frac{\frac{1 - {G'_l}^2}{\sqrt{G'_l}}}{1 + G'_l} =\\
    &= \pm \frac{1 -{G'_l}^2 }{\sqrt{G'_l} (1 +{G'_l})} = \pm \frac{(1 -{G'_l})\cancel{(1 +{G'_l})}}{\sqrt{G'_l} \cancel{(1 +{G'_l})}} = \pm \frac{1 - {G'_l}}{\sqrt{G'_l}}
    \end{aligned}
\end{equation*}
Quindi:
\begin{equation*}
    B_p = B'_p \cdot Y_0 = \pm \frac{1 - \frac{G_l}{Y_0}}{\sqrt{\frac{G_l}{Y_0}}} \cdot Y_0 = \pm \frac{Y_0 - G_l}{\sqrt{G_l} / \sqrt{Y_0}} = \pm (Y_0 - G_l) \cdot \sqrt{\frac{Y_0}{G_l}}
\end{equation*}
Questa \textbf{reattanza} va \textbf{compensata} con una \textbf{reattanza} \textbf{uguale} e \textbf{opposta} attraverso uno \textbf{stub} \textbf{parallelo}:
\begin{equation*}
    B_{stub} = Y_0 co\tan(kl_{stub}) = - B_p
\end{equation*}
\begin{equation*}
    kl_{stub} = -arcotan\left(\frac{B_p}{Y_0}\right) + n\pi
\end{equation*}
\begin{equation*}
   \implies l_{stub} = -arcotan\left(\frac{B_p}{Y_0}\right) \frac{\lambda}{2\pi} + n \frac{\lambda}{2} \geq 0
\end{equation*}


\section{Adattamento Tramite Inverter}
Consideriamo la seguente \textbf{linea senza perdite chiusa su un carico}:
\begin{center}
    \includegraphics[width=.6\textwidth]{Images/figure43.png}
\end{center}
L'impedenza $Z'_L$ sarà:
\begin{equation*}
    Z'_L = Z_T \frac{Z_L + j Z_T \tan(k_T l)}{Z_T + j Z_L \tan(k_T l)}
\end{equation*}

Nel caso in cui:
\begin{equation*}
    k_T l = \frac{\pi}{2} \implies l = \frac{\lambda_T}{4}
\end{equation*}
Allora:
\begin{equation*}
    k_T l = \left(\frac{\pi}{2}\right) \longrightarrow \infty
\end{equation*}
Quindi va calcolata con il limite:
\begin{equation*}
    Z'_L = \lim\limits_{\tan(k_T l) \longrightarrow \infty} Z_T \frac{Z_L + j Z_T \tan(k_T l)}{Z_T + j Z_L \tan(k_T l)} = \frac{Z_T^2}{Z_L} = Z_T^2 Y_L
\end{equation*}
\textbf{Quindi possiamo dire che il trasporto di un'impedenza di un quarto di lunghezza d'onda lungo una linea da luogo ad un'impedenza proporzionale al reciproco dell'impedenza originale.}




\chapter{Allocazione delle Risorse}
In questo capitolo ci occuperemo di trasportare gran parte delle nozioni di ottimizzazione  apprese nel capitolo precedente, alle Reti Wireless.
\\
Per prima cosa dobbiamo chiederci se le ipotesi per usare la teoria dell'ottimizzazione concava/pseudo-concava sono verificate:
\\ \\
Per il Sum-Rate e per la GEE abbiamo bisogno che i rate/EE di ciascun utente siano concavi :
    \begin{equation*}
        SR = \sum_{k=1}^K \log(1 + sinr_k) 
    \end{equation*}
    \begin{equation*}
         GEE = \frac{\sum_{k=1}^K \log(1 + sinr_k)}{P_c + \sum_{k=1}^K \mu_k p_k}
    \end{equation*}


Tuttavia ricordando quali siano le formule del rate e dell'efficienza energetica in una rete SIMO in uplink:
\begin{equation*}
    \begin{aligned}
    R_m &= \log\left(1 + \frac{p_m |\vc{c}_m^H \vc{h}_m |^2}{\sigma^2 + \color{red} \sum_{k\neq m} p_l |\vc{c}_m^H \vc{h}_k |^2} \right) \\ \\
        EE_m &= \frac{\log\left(1 + \frac{p_m |\vc{c}_m^H \vc{h}_m |^2}{\sigma^2 + \color{red} \sum_{k\neq m} p_l |\vc{c}_m^H \vc{h}_k |^2} \right)}{\mu_m p_m + P_{c,m}}
    \end{aligned}
\end{equation*}
Mentre nel downlink abbiamo:
\begin{equation*}
    \begin{aligned}
    R_m &= \log\left(1 + \frac{p_m |\vc{g}_m^H \vc{q}_m |^2}{\sigma^2 + \color{red} \sum_{k\neq m} p_l |\vc{g}_m^H \vc{q}_k |^2} \right) \\ \\
        EE_m &= \frac{\log\left(1 + \frac{p_m |\vc{g}_m^H \vc{q}_m |^2}{\sigma^2 + \color{red} \sum_{k\neq m} p_l |\vc{g}_m^H \vc{q}_k |^2} \right)}{\mu_m p_m + P_{c,m}}
    \end{aligned}
\end{equation*}
Dove la parte rossa rappresenta l'Interferenza Multi-Utente, che rende il rate una funzione non-concava e di conseguenza neanche l'efficienza energetica...\\ \\
\begin{center}
    Allora l'unica cosa da fare è eliminare l'Interferenza Multi-Utente.
\end{center}
Possiamo fare ciò attraverso diverse strade:
\begin{itemize}
    \item Usando un ricevitore/trasmettitore di tipo Zero-Forcing.
    \item Separando gli utenti in frequenza/tempo (FDMA/TDMA).
    \item Possiamo Considerare un sistema massive MIMO.
\end{itemize}
\section{Uplink Power Control}
Supponendo di aver tolto l'Interferenza Multi-Utente, i nostri problemi si ridurranno nei seguenti:
\subsection{Uplink Sum Rate Maximization}
\begin{equation*}
    \begin{aligned}
    &\max_{\vc{p}} \sum_{m=1}^K \log\left(1 + \frac{p_m |\vc{c}_m^H \vc{h}_m |^2}{\sigma^2} \right) \\
     &con  \ 0 \leq p_m \leq P_{max,m}, \ \forall m
     \end{aligned}
\end{equation*}
Dato che ciascun membro della sommatoria e ciascun vincolo dipendono solo da $p_m$, possiamo far entrare max dentro la sommatoria:
\begin{equation*}
    \begin{aligned}
     &\max_{p_m}\log\left(1 + \frac{p_m |\vc{c}_m^H \vc{h}_m |^2}{\sigma^2} \right) \\
     & con  \ 0 \leq p_m \leq P_{max,m}, \ \forall m
    \end{aligned}
\end{equation*}
Allora il risultato è:
\begin{equation*}
    p_m = P_{max,m}
\end{equation*}

\subsection{GEE Maximization}
\begin{equation*}
    \begin{aligned}
    &\max_{\vc{p}} \frac{\sum_{m=1}^K \log\left(1 + \frac{p_m |\vc{c}_m^H \vc{h}_m |^2}{\sigma^2} \right)}{\mu_m p_m + P_{c,m}} \\
    &con \ 0 \leq p_m \leq P_{max,m}, \ \forall m
    \end{aligned}
\end{equation*}
Qui siamo costretti ad usare l'algoritmo di Dinkelbach, risolvendo in ciascuna iterazione il seguente problema ausiliario:
\begin{equation*}
    \begin{aligned}
    &\max_{\vc{p}} \sum_{m=1}^K \log\left(1 + \frac{p_m |\vc{c}_m^H \vc{h}_m |^2}{\sigma^2} \right) - \lambda(\mu_m p_m + P_{c,m})
    \\
    &con \ 0 \leq p_m \leq P_{max,m}, \ \forall m
    \end{aligned}
\end{equation*}
\begin{center}
    Adesso possiamo far entrare il max dentro la sommatoria!
\end{center}
\begin{equation*}
    \begin{aligned}
     &\max_{p_m}\log\left(1 + \frac{p_m |\vc{c}_m^H \vc{h}_m |^2}{\sigma^2} \right) - \lambda(\mu_m p_m + P_{c,m}) \\
     & con  \ 0 \leq p_m \leq P_{max,m}, \ \forall m
    \end{aligned}
\end{equation*}
Ora definendo:
\begin{equation*}
    a_m = |\vc{c}_m^H \vc{h}_m |^2 / \sigma^2
\end{equation*}
Studiamo il segno della derivata prima:

\begin{equation*}
    \begin{aligned}
    &\frac{a_m}{ln(2) (a_m p_m +1)} - \lambda \mu_m \geq 0 \implies \\
    &\implies p_m \leq \left(\frac{1}{ln(2)\lambda \mu_m } - \frac{1}{a_m}\right) = \Tilde{p_m}
    \end{aligned}
\end{equation*}

Quindi la funzione obbiettivo è:
\begin{itemize}
    \item Crescente per $p_m \leq \Tilde{p_m}$
    \item Decrescente per $p_m \geq \Tilde{p_m}$
    \item Con un punto di massimo in $p_m = \Tilde{p_m}$
\end{itemize}  
Quindi considerando anche i vincoli avremo la seguente soluzione:
\begin{equation*}
    p_m^* = max\{0, min\{P_{max,m},\Tilde{p_m} \}\}
\end{equation*}


\subsection{Downlink Sum Rate Maximization}
Vale un discorso analogo per il Downlink:
\begin{equation*}
    \begin{aligned}
    &\max_{\vc{p}} \sum_{m=1}^K \log\left(1 + \frac{p_m |\vc{g}_m^H \vc{q}_m |^2}{\sigma^2} \right) \\
     &con  \ 0 \leq p_m \leq P_{max,m}, \ \forall m \\
     &\sum_{m=1}^K p_m \leq P_{max}
     \end{aligned}
\end{equation*}
Dato che la funzione obbiettivo, all'aumentare di $p_m$ cresce, vuol dire che la soluzione del problema si avrà per $p_m = p_{max}$, quindi:
\begin{equation*}
    \begin{aligned}
    &\max_{\vc{p}} \sum_{m=1}^K \log\left(1 + \frac{p_m |\vc{g}_m^H \vc{q}_m |^2}{\sigma^2} \right) \\
     &con  \ 0 \leq p_m \leq P_{max,m}, \ \forall m \\
     &\sum_{m=1}^K p_m = P_{max}
     \end{aligned}
\end{equation*}
Che non è altro che un problema di Water-Filling con:
\begin{equation*}
     c_m = |\vc{g}_m^H \vc{q}_m |^2 / \sigma^2
\end{equation*}
quindi la soluzione è:
\begin{equation*}
    p_m = max\left(\frac{1}{\nu ln(2)} - \frac{1}{c_m}\right)
\end{equation*}

\subsection{Downlink GEE Maximization}
\begin{equation*}
    \begin{aligned}
    &\max_{\vc{p}} \frac{\sum_{m=1}^K \log\left(1 + \frac{p_m |\vc{g}_m^H \vc{q}_m |^2}{\sigma^2 } \right)}{\mu_m p_m + P_{c,m}} \\
    &con \ 0 \leq p_m \leq P_{max,m}, \ \forall m \\
     &\sum_{m=1}^K p_m \leq P_{max}
    \end{aligned}
\end{equation*}
Dobbiamo applicare, come nell'uplink, l'algoritmo di Dinkelbach, dove in ciascuna iterazione abbiamo il seguente problema ausiliario:
\begin{equation*}
    \begin{aligned}
    &\max_{\vc{p}} \sum_{m=1}^K  \log\left(1 + \frac{p_m |\vc{g}_m^H \vc{q}_m |^2}{\sigma^2 } \right) - \lambda \mu p_m \\
    &con \ p_m \geq 0 \ \forall m \\
    &\sum_{m=1}^K p_m \leq P_{max}
    \end{aligned}
\end{equation*}
Purtroppo qui non possiamo mettere l'uguaglianza nell'ultimo vincolo dato che la nostra funzione obiettivo non è crescente in $p_m$, quindi la somma delle Potenze Ottime sarà:
\begin{equation*}
    \sum_{m=1}^K p_m^* = P \leq P_{max}
\end{equation*}
Quindi dobbiamo risolvere il problema ausiliario per ogni $P \in [0, P_{max}]$:
\begin{equation}
    \begin{aligned}
     &\max_{\vc{p}} \sum_{m=1}^K  \log\left(1 + \frac{p_m |\vc{g}_m^H \vc{q}_m |^2}{\sigma^2 } \right) - \lambda \mu p_m \\
    &con \ p_m \geq 0 \ \forall m \\
    &\sum_{m=1}^K p_m = P
    \end{aligned}
\end{equation}
\begin{center}
    E poi scegliere la soluzione corrispondente alla P con la quale otteniamo il valore più grande della funzione obiettivo.
\end{center}
\begin{algorithm}
\begin{algorithmic}
\Large
\State {Inizializza} $\delta > 0; P_{tot} = [0 : \delta : P_{max}];$
\State \textbf{for} n = 1 : length(P) \textbf{do}
\State $P = P_{tot}(n);$
\State $Solve \ (6.1) \ to \ obtain  \ \vc{p}_n^*;$
\State $F(n) = \sum_{m = 1}^K \log\left(1 + \frac{p_{m,n}^* |\vc{g}_m^H \vc{q}_m |^2}{\sigma^2 } \right) - \lambda \mu p_{m,n}^*$
\State \textbf{end for}
\State $l = argmax F(n);$
\State $\vc{p}^* = \vc{p}_l^*;$
\end{algorithmic}
\end{algorithm}
\begin{center}
    Quindi dobbiamo risolvere 6.1 in ogni iterazione!
\end{center}
Definendo:
\begin{equation*}
    c_m = |\vc{g}_m^H \vc{q}_m |^2 / \sigma^2
\end{equation*}
Calcoliamo il Lagrangiano della funzione obiettivo:
\begin{equation*}
    \begin{aligned}
    L(\vc{x}, \vc{\psi}, \vc{nu}) &= - \sum_{m=1}^K log(1 + c_m p_m) + \lambda \mu p_m - \\
    & - \sum_{m=1}^K \psi_m p_m + \nu \left(\sum_{m=1}^K p_m - P\right)
    \end{aligned}
\end{equation*}
Ora poniamo la derivata di L rispetto alla generica $p_k$ uguale a 0:
\begin{equation*}
    \frac{\partial L}{\partial p_k} = \frac{-c_k / ln(2)}{1 + c_k p_k} + \lambda \mu - \psi_k + \nu = 0
\end{equation*}
Quindi definendo $\nu_{eq} = \nu + \lambda \mu$ otteniamo:
\begin{equation*}
    \psi_k = \nu_{eq} - \frac{c_k /ln(2)}{1 + c_k p_k}
\end{equation*}
\begin{center}
    Quindi la soluzione è:
\end{center}
\begin{equation*}
    p_k = max\left(0, \frac{1}{\nu_{eq} ln(2)} - \frac{1}{c_k}\right)
\end{equation*}


\section{MIMO Maximization}
\subsection{MIMO Rate Maximization}
Consideriamo ora un sistema MIMO a singolo utente, in cui dopo aver ottimizzato:
\begin{itemize}
    \item La Matrice di Beamforming $\vc{U}_Q$
    \item La Matrice di Ricezione $\vc{C}$
\end{itemize}
la Capacità si scrive così:
\begin{equation*}
    \sum_{i=1}^{min\{N_R, N_T\}} C_i = B \sum_{i=1}^{min\{N_R, N_T\}} log\left(1 + \frac{p}{\sigma^2} \lambda_{i,H}^2 \lambda_{i,Q}\right)
\end{equation*}
Allora in questo caso il problema da risolvere è:
\begin{equation*}
    \begin{aligned}
    & max_{\vc{\lambda_Q}}  \sum_{i=1}^{min\{N_R, N_T\}} log\left(1 + \frac{p}{\sigma^2} \lambda_{i,H}^2 \lambda_{i,Q}\right) \\
    & con \ \lambda_{i,Q} \geq 0, \ \forall i \\
    & \sum_{i=1}^{min\{N_R, N_T\}} \lambda_{i,Q} \leq 1
    \end{aligned}
\end{equation*}
Ma possiamo riformulare il problema così:
\begin{equation*}
    \begin{aligned}
    & max_{p,\vc{\lambda_Q}}  \sum_{i=1}^{min\{N_R, N_T\}} log\left(1 + \frac{p}{\sigma^2} \lambda_{i,H}^2 \lambda_{i,Q}\right) \\
    & con \ \lambda_{i,Q} \geq 0, \ \forall i \\
    & \sum_{i=1}^{min\{N_R, N_T\}} \lambda_{i,Q} = 1 \\
    & p \in [0, P_{max}]
    \end{aligned}
\end{equation*}
Dunque, abbiamo un problema di tipo Water-Filling, con:
\begin{equation*}
    c_i = \frac{p}{\sigma^2} \lambda^2_{i, H}
\end{equation*}
Quindi la soluzione sarà:
\begin{equation*}
  \lambda_{i,Q} =  max \left(0, \frac{1}{\nu} - \frac{1}{c_i}\right)
\end{equation*}

\subsection{MIMO EE Maximization}
In questo caso il problema è:
\begin{equation*}
    \begin{aligned}
    & max_{p,\vc{\lambda_Q}} \frac{\sum_{i=1}^{min\{N_R, N_T\}} log\left(1 + \frac{p}{\sigma^2} \lambda_{i,H}^2 \lambda_{i,Q}\right)}{\mu p + P_c}  \\
    & con \ 0 \leq p \leq P_{max}
    \end{aligned}
\end{equation*}

Gli autovalori $\lambda_{i,Q}$, come si può vedere, influenzano solo il Rate, quindi possono essere allocati come nella MIMO Rate Maximization.\\ \\
Dato che rispetto a p, la GEE è strettamente pseudo-concava, definendo con $\hat{p}$ il punto stazionario, la soluzione è:
\begin{equation*}
    \hat{p} = max \{0, min\{P_{max}, \hat{p}\}\}
\end{equation*}


\chapter{Convertitori A/D}
\section{Voltmetro Numerico}
Il \textbf{Voltmetro Numerico} è uno strumento che effettua misure di \textbf{tensione} mediante una \textbf{conversione analogico-digitale} della grandezza in ingresso e che visualizza il risultato numerico su un \textbf{display}:
\begin{center}
    \includegraphics[width=\textwidth]{Images/figure25.png}
\end{center}
\begin{itemize}
    \item \textbf{Sezione di Condizionamento del Segnale in ingresso}, costituito da \textbf{partitori resistivi}, che ha il compito di amplificare/attenuare/filtrare il segnale per i blocchi successivi;
    \item \textbf{ADC} (Convertitore Analogico Digitale);
    \item \textbf{Microprocessore};
    \item \textbf{Display};
\end{itemize}
In particolare il \textbf{Convertitore Analogico Digitale} si può dividere in:
\begin{itemize}
    \item Convertitori \textbf{Istantanei};
    \item Convertitori ad \textbf{Integrazione};
\end{itemize}
Quelli \textbf{Istantanei} sono in genere più \textbf{veloci}, approssimando il segnale attraverso una\textbf{ serie di gradini}.\\ \\
Quelli ad \textbf{Integrazione} invece vengono usati per misurare \textbf{tensione continua} grazie alla loro elevata \textbf{reiezione al rumore }sovrapposto al segnale, tuttavia hanno un \textbf{tempo di conversione elevato}.
\section{Convertitore Multi Rampa}
\begin{center}
    \includegraphics[width=.8\textwidth]{Images/figure46.png}
\end{center}
L'idea generale di questo \textbf{convertitore} è di fare una \textbf{serie di misure}, da prima \textbf{grossolane} e sempre via via più \textbf{precise}.

\section{Convertitore a Doppia Rampa}
Questo tipo di convertitore è utilizzato nel \textbf{multimetro} e consiste in una conversione \textbf{tensione/tempo}.\\ \\
Lo scherma a blocchi è costituito da:
\begin{itemize}
    \item Un \textbf{Integratore};
    \item Un \textbf{Comparatore di Zero};
    \item Un \textbf{Generatore di Tensione di Riferimento};
    \item Un Circuito per la \textbf{Misura dei Tempi}
\end{itemize}
\begin{center}
    \includegraphics[width=\textwidth]{Images/figure27.png}
\end{center}

La misura avviene in due modi distinti:
\begin{itemize}
    \item Il segnale da misurare viene applicato all'ingresso del circuito e si inizia la \textbf{fase di conversione}; L'Integratore provvede ad \textbf{integrare} la tensione in ingresso per un \textbf{intervallo costante di tempo} ($T_{up}$);
    \item Dopo il tempo $T_{up}$, il \textbf{circuito di controllo} commuta l'ingresso dell'integratore verso una \textbf{tensione di riferimento} d'ampiezza nota e segno opposto alla tensione da misurare. A questo punto avviene una \textbf{seconda fase di integrazione} in cui la tensione si riduce a \textbf{zero}. Quando ciò avviene, il \textbf{comparatore} segnala il \textbf{passaggio per} \textbf{lo} \textbf{zero} e la Logica di Controllo interrompe il conteggio degli impulsi.
\end{itemize}
\begin{equation*}
    V_x = E_c \frac{T_{up}}{T_{down}}
\end{equation*}
\section{Convertitore ad Approssimazioni Successive}
\begin{center}
    \includegraphics[width=.4\textwidth]{Images/figure42.png}
\end{center}
In questo dispositivo, un'opportuna catena di reazione fa \textbf{variare il D/A} fin quando non \textbf{eguaglia} la tensione del segnale analogico che si vuole misurare.\\
Il controllo del D/A avviene tramite un \textbf{registro di approssimazioni successive} (SAR) che pone a \textbf{1 il bit più significativo} (MSB).\\
Il \textbf{comparatore} confronta \textbf{l'uscita del D/A} con \textbf{l'ingresso analogico} e lo lascia ad 1 se D/A $<$ Ingresso oppure 0 se D/A $>$ ingresso. (\textbf{Ricerca in un albero di ricerca)}\\
Dopo N confronti, il \textbf{SAR} confronterà il valore numerico corrispondente all'ingresso analogico.
\section{Convertitore di Tipo parallelo (Flash)}
\begin{center}
    \includegraphics[width=.5\textwidth]{Images/figure45.png}
\end{center}
Questo convertitore è il più \textbf{veloce} tra quelli esistenti, esso è costituito da $2^n - 1$ \textbf{contatori} \textbf{analogici} che confrontano il segnale con $2^n -1$ valori diversi di \textbf{tensione} \textbf{di} \textbf{riferimento}.\\ \\
Il vantaggio maggiore è dovuto al fatto che tutti i \textbf{comparatori} eseguono il confronto nello \textbf{stesso} \textbf{istante}, riducendo il processo di \textbf{quantizzazione} ad un solo istante.\\ \\
Ci sono diverse sorgenti di \textbf{Errore}:
\begin{itemize}
    \item \textbf{Errori} \textbf{Statici} introdotti dai \textbf{comparatori} sia come tensioni di offset sia come correnti di offset e di bias;
    \item \textbf{Errori} dovuti alla \textbf{rete} \textbf{resistiva};
    \item \textbf{Errori} \textbf{Dinamici} dovuti ai \textbf{comparatori}
\end{itemize}

\section{Convertitore Serie Parallelo (Pipeline)}
\begin{center}
    \includegraphics[width=\textwidth]{Images/figure39.png}
\end{center}
Il segnale di ingresso viene \textbf{convertito in digitale} in un primo stadio, \textbf{riconvertito in analogico} in un secondo e successivamente viene \textbf{sottratto} con se stesso.\\ \\
Il risultato è ottenuto dopo una moltiplicazione per $2^n$, quindi il risultato, composto dall'insieme dei bit ottenuti, dovrà essere \textbf{shiftato}.\\ 
Questa misura \textbf{dura un colpo di clock}.

\chapter{Convertitori D/A}
\section{Convertitore a Resistenze Pesate}
\begin{center}
    \includegraphics[width=\textwidth]{Images/figure37.png}
\end{center}
Questo convertitore presenta un \textbf{insieme di resistori di valore multiplo} (potenza di 2) di un valore \textbf{R}.
\begin{equation*}
    V_{out} = \sum \frac{V_i}{2^i} \cdot V_R
\end{equation*}
Le \textbf{resistenze} sono alimentate da una \textbf{tensione di riferimento} all'ingresso dell'\textbf{amplificatore} \textbf{operazionale}, che a sua volta in ingresso presenta una \textbf{resistenza infinita}, per cui tutta la corrente che arriva in ingresso arriva in uscita.\\ \\
In \textbf{ingresso abbiamo i bit, che pilotano gli interruttori}.\\
(1 chiuso, 0 aperto)
\section{Convertitore R-2-R}
\begin{center}
    \includegraphics[width=\textwidth]{Images/figure38.png}
\end{center}
In questo convertitore, facendo una banale osservazione di \textbf{serie} e \textbf{paralleli}, possiamo dire che la \textbf{corrente} è:
\begin{equation}
    I_{in}= \frac{V_R}{2R}
\end{equation}
In particolare, la \textbf{caratteristica ingresso-uscita} di questo convertitore è la \textbf{stessa} del convertitore a \textbf{resistenze pesate}:
\begin{equation}
   V_{out} =  I_{in} R= \frac{V_R}{2}
\end{equation}

\chapter{FFT Analyzer}
La \textbf{FFT} (Fast Fourier Transform) non è altro che un \textbf{algoritmo} che calcola più velocemente la \textbf{DFT} (Discrete Fourier Transform):
\begin{center}
    \includegraphics[width=\textwidth]{Images/figure30.png}
\end{center}
Il \textbf{segnale analogico} entra nel \textbf{blocco di condizionamento}, a valle del quale avremo la \textbf{conversione analogico digitale} che campiona il segnale ad una \textbf{frequenza di campionamento} $f_c$.\\
Il segnale viene poi immagazzinato in \textbf{memoria} e \textbf{finestrato}, e a questo punto viene effettuata la \textbf{FFT}.\\ \\
Vediamo i blocchi in dettaglio:\\
\begin{center}
    \textbf{Condizionamento}
\end{center}
Questo blocco ha lo scopo di \textbf{adattare il segnale analogico} per i blocchi che troverà a valle e opera la funzione di un \textbf{filtro} \textbf{passa basso}, con l'aggiunta della \textbf{protezione dal fenomeno di aliasing}.\\ \\
In particolare, il compito di questo filtro è quello di \textbf{limitare le componenti frequenziali} del segnale in ingresso, in modo tale da far si che la \textbf{massima frequenza }del segnale sia \textbf{minore della metà della frequenza di campionamento}.\\ \\
Entrando più nel dettaglio possiamo discernere \textbf{3 operazioni fondamentali}:
\begin{itemize}
    \item \textbf{Accoppiamento} (coupling)
    \item \textbf{Amplificatore}/\textbf{Attenuatore}
    \item \textbf{Anti-Aliasing}
\end{itemize}
\begin{center}
    \textbf{Memoria}
\end{center}
Identicamente alla \textbf{memoria dell'oscilloscopio}, anche la memoria dell' \textbf{FFT} viene gestita con la tecnica \textbf{FIFO}.\\ \\
\begin{center}
    \textbf{Finestratura}
\end{center}
Il calcolo della \textbf{FFT} si basa su \textbf{sequenze} di lunghezza N finite, mentre arrivano in continuazione campioni in uscita dall'A/D, quindi occorre selezionare dei blocchi di N campioni per volta.\\ \\
Questa operazione viene chiamata \textbf{finestratura} e consiste nel \textbf{moltiplicare} la sequenza di uscita per una \textbf{finestra} \textbf{rettangolare di lunghezza N}.\\ \\
Possiamo definire due tipi di campionamenti:
\begin{itemize}
    \item \textbf{Campionamento Sincrono}, in cui prendiamo un numero intero di periodi;
    \item \textbf{Campionamento Asincrono}, in cui non prendiamo un numero intero di periodi;
\end{itemize}
Nel caso di \textbf{campionamento asincrono}, si genereranno delle \textbf{discontinuità} che faranno nascere delle \textbf{componenti spettrali} non presenti nel segnale originario (\textbf{Spectral} \textbf{Leakage}).\\ \\
Proprio per questo esistono diverse tipologie di finestre:
\begin{itemize}
    \item \textbf{Rettangolare}
    \item \textbf{Hamming}
    \item \textbf{Blackman-Harris}
    \item \textbf{Flat Top}
\end{itemize}
\begin{center}
    \includegraphics[width=.8\textwidth]{Images/figure47.png}
\end{center}
\chapter{Parametri S di un Doppio Bipolo}
\begin{center}
    \includegraphics[width=.8\textwidth]{Images/figure38.png}
\end{center}
Scegliamo le \textbf{ampiezze delle onde incidenti} come \textbf{variabili indipendenti}:
\begin{squared}[violet]
    \begin{dcases}
    V_1^- = S_{11} V_1^+ + S_{12} V_2^+\\
    V_2^- = S_{21} V_1^+ + S_{22} V_2^+
    \end{dcases}
    \implies \underline{V}^- = \underline{\underline{S}} \underline{V}^+
\end{squared}
I coefficienti $S_{i,j}$ sono \textbf{adimensionali} e detti \textbf{parametri di Scattering}.
\section{Reciprocità}
Dividiamo due casi:
\begin{center}
    \textbf{(a)}\\
    \includegraphics[width=.7\textwidth]{Images/figure39.png}
\end{center}
\begin{equation*}
    I_{01}^a = 2 I_1^a = 2 \frac{V_1^{a+}}{Z_0}
\end{equation*}
\begin{center}
    \textbf{(b)}\\
    \includegraphics[width=.7\textwidth]{Images/figure40.png}
\end{center}
\begin{equation*}
    I_{02}^b = 2 I_1^b = 2 \frac{V_2^{b+}}{Z_0}
\end{equation*}
Il \textbf{teorema di reciprocità} applicato in questo caso ci da:
\begin{equation*}
    I_{01}^a V_1^b = I_{02}^b V_2^a
\end{equation*}
Che sostituendo le \textbf{correnti} diventa:
\begin{equation*}
\tag{x}
    V_1^{a+} V_1^b =V_2^{b+} V_2^a
\end{equation*}
Calcoliamo $ V_2^a$:
\begin{equation*}
     V_2^a = \cancel{V_2^{a+}} + V_2^{a-} = V_2^{a-} = S_{21} V_1^{a+} + \cancel{S_{22} V_2^{a+}}
\end{equation*}
\begin{equation*}
    \implies V_2^a = S_{21} V_1^{a+}
\end{equation*}
Calcoliamo $ V_1^b$:
\begin{equation*}
     V_1^b = \cancel{V_1^{b+}} + V_1^{b-} = V_1^{b-} = \cancel{S_{12} V_1^{b+}} + S_{12} V_2^{b+}
\end{equation*}
\begin{equation*}
    \implies V_1^b = S_{12} V_2^{b+}
\end{equation*}
Sostituiamo in (x):
\begin{equation*}
     \cancel{V_1^{a+}} S_{12} \cancel{V_2^{b+}} = \cancel{V_2^{b+}} S_{21} \cancel{V_1^{a+}}
\end{equation*}
\begin{squared}
    \implies S_{12} = S_{21} \implies \underline{\underline{S}} = \underline{\underline{S}}^T
\end{squared}

\section{Senza Perdite}
\textbf{Assenza di perdite} vuol dire che la \textbf{somma delle potenze attive alle due porte è nulla}.\\
Possiamo calcolare la potenza alle due porte tramite la \textbf{differenza della potenza incidente e quella riflessa}, quindi:
\begin{equation*}
    P_{diss} = \frac{1}{2} \frac{|V_1^+|^2}{Z_0} - \frac{1}{2} \frac{|V_1^-|^2}{Z_0} + \frac{1}{2} \frac{|V_2^+|^2}{Z_0} - \frac{1}{2} \frac{|V_2^-|^2}{Z_0} = 0
\end{equation*}
\begin{equation*}
    \implies \cancel{\frac{1}{2}} \frac{|V_1^+|^2}{\cancel{Z_0}} + \cancel{\frac{1}{2}} \frac{|V_2^+|^2}{\cancel{Z_0}} = \cancel{\frac{1}{2}} \frac{|V_1^-|^2}{\cancel{Z_0}} + \cancel{\frac{1}{2}} \frac{|V_2^-|^2}{\cancel{Z_0}}
\end{equation*}
\begin{equation*}
    \implies |V_1^+|^2 + |V_2^+|^2 = |V_1^-|^2 + |V_2^-|^2
\end{equation*}
In \textbf{forma matriciale} sarebbe:
\begin{equation*}
    \begin{bmatrix}
    V_1^{+^*}, V_2^{+^*}
    \end{bmatrix}
    \begin{bmatrix}
    V_1^{+}\\ V_2^{+}
    \end{bmatrix}
    =
    \begin{bmatrix}
    V_1^{-^*}, V_2^{-^*}
    \end{bmatrix}
    \begin{bmatrix}
    V_1^{-}\\ V_2^{-}
    \end{bmatrix}
\end{equation*}
\begin{equation*}
    {\underline{V}}^{+^H} {\underline{V}}^+ = {\underline{V}}^{-^H} {\underline{V}}^-
\end{equation*}
\begin{equation*}
    {\underline{V}}^{+^H} {\underline{V}}^+ = \left(\underline{\underline{S}}{\underline{V}}^{+} \right)^H \left(\underline{\underline{S}}{\underline{V}}^{+} \right)
\end{equation*}
\begin{equation*}
    \implies \underline{{V}}^{+^H} {\underline{V}}^{+} = {\underline{V}}^{+^H} \left(\underline{\underline{S}}^H\underline{\underline{S}}\right){\underline{V}}^{+}
\end{equation*}
Dato che deve valere che $\forall  \ \underline{V}^{+}  $
\begin{equation*}
    \implies \underline{\underline{S}}^H\underline{\underline{S}} = \underline{\underline{I}}
\end{equation*}
Analizziamo ora elemento per elemento $\underline{\underline{S}}$ per \textbf{verificare se rispetta la condizione}:
\begin{equation*}
    \begin{bmatrix}
   S_{11}^* & S_{12}^* \\
   S_{21}^* & S_{22}^*
    \end{bmatrix}
    \cdot 
    \begin{bmatrix}
   S_{11} & S_{12} \\
   S_{21} & S_{22}
    \end{bmatrix}
    =
    \begin{bmatrix}
   1 & 0 \\
   0 & 1
    \end{bmatrix}
\end{equation*}
Quindi:
\begin{equation*}
    \begin{dcases}
    |S_{11}|^2 + |S_{12}|^2 = 1 \\
    S_{12}^*S_{11} + S_{22}^* S_{12} = 0\\
    S_{11}^*S_{12} + S_{12}^* S_{22} = 0\\
    |S_{12}|^2 + |S_{22}|^2 = 1
    \end{dcases}
\end{equation*}
Ricordando che $S_{12} = S_{21}$:
\begin{equation*}
    |S_{22}|^2 = 1 - |S_{12}|^2 ,\quad |S_{21}|^2 = 1 -  |S_{22}|^2 \implies |S_{22}|^2 = |S_{11}|^2 = s^2
\end{equation*}
Da cui:
\begin{squared}
\begin{dcases}
    S_{11} = s e^{j \varphi_{11}}\\
    S_{22} = s e^{j \varphi_{22}}\\
    S_{12} = S_{21} = \sqrt{1 - s^2} e^{j \varphi_{12}}
\end{dcases}
\end{squared}
Sostituiamo in $S_{11}^*S_{12} + S_{21}^* S_{22} = 0$:
\begin{equation*}
    s e^{-j \varphi_{11}} \sqrt{1 - s^2} e^{-j \varphi_{12}} + \sqrt{1 - s^2} e^{-j \varphi_{12}} s e^{-j\varphi_{22}} = 0
\end{equation*}
Da cui:
\begin{equation*}
    e^{-j \varphi_{11}}e^{j \varphi_{12}}+e^{-j \varphi_{12}}e^{j \varphi_{22}} = 0
\end{equation*}
Ovvero:
\begin{equation*}
    \begin{aligned}
    &e^{j(\varphi_{12} - \varphi_{11})} = -e^{j(\varphi_{22}-\varphi_{12})}\\
    &e^{j(\varphi_{12} - \varphi_{11})} = e^{j(\varphi_{22}-\varphi_{12} + \pi)}\\
    &\varphi_{12} - \varphi_{11} = \varphi_{22}+\varphi_{12} + \pi + 2n\pi\\
    &2\varphi_{12} = \varphi_{22}+\varphi_{11}+ \pi + 2n\pi
    \end{aligned}
\end{equation*}
\begin{squared}
    \varphi_{12} = \frac{\varphi_{22} + \varphi_{11}}{2}  \pm \frac{\pi}{2}
\end{squared}
\chapter{Q-Metro}
Il \textbf{Q-Metro} misura le \textbf{impedenze} e i \textbf{fattori di merito/perdita} e si basa sul principio di \textbf{risonanza}:
  \begin{center}
    \includegraphics[width=0.40\textwidth]{Images/figure23.png}
  \end{center}
\begin{equation*}
    \w_0 L = \frac{1}{\w_0 C} \implies \w_0 = \sqrt{\frac{1}{LC}}
\end{equation*}
\begin{equation*}
    V_C = \underbrace{\frac{E}{R}}_{I} \cdot \frac{1}{\w_0 C} = E \frac{\w_0 L}{R} = E \cdot Q
\end{equation*}
Si dimostra che $\w_{max} = \w_0 \sqrt{1 - \frac{1}{2Q}}$ e se $Q > 10$ siamo in risonanza.\\ 
\section{Q-Metro con sostituzione tipo serie}
\begin{center}
    \includegraphics[width=\textwidth]{Images/figure24.png}
\end{center}
Dove $R_s$ è una resistenza molto piccola che serve per misurare E.\\ \\ \\ \\
\begin{center}
    \textbf{Prima misurazione (senza $Z_x$)}
\end{center}
\begin{equation*}
    X_L = X_{C_1} 
\end{equation*}
Che va in risonanza quando:
\begin{equation*}
    Q_1 = \frac{\w L}{R} = \frac{1}{\w R C_1} \implies R = \frac{1}{\w Q_1 C_1}
\end{equation*}
\begin{center}
    \textbf{Seconda misurazione (con $Z_x$)}
\end{center}
\begin{equation*}
    X_L + X_x = X_{C_2} \implies X_x = X_{C_2} - X_{C_1} = \frac{1}{\w C_2} - \frac{1}{\w C_1} = \frac{C_1 - C_2}{\w C_1 C_2} \implies
\end{equation*}
\begin{equation*}
    \implies Q_2 = \frac{1}{\w (R + R_x) C_2} \implies R+ R_x = \frac{1}{\w Q_2 C_2} \implies
\end{equation*}
\begin{equation*}
    \implies R_x = \frac{1}{\w Q_2 C_2} - \frac{1}{\w Q_1 C_1}
\end{equation*}
Da notare che, se $C_1 \approx C_2$ non va bene...\\
Va bene invece se $X_x$ ha $L >> 1$ oppure $C << 1$.
\chapter{Impedenzimetro}
\begin{center}
    \includegraphics[width=\textwidth]{Images/figure40.png}
\end{center}
Questo strumento serve per misurare \textbf{R, L, C, Q e sfasamenti}, e si basa su una tecnica di tipo \textbf{voltamperometrica}.\\ \\
Esso è composto da:
\begin{itemize}
    \item Un \textbf{generatore di segnale}
    \item \textbf{Impedenza incognita} $Z_x$
    \item \textbf{Resistenza Campione }$Z_x$
    \item \textbf{Amplificatori differenziali} $A_1, A_2$
    \item \textbf{Amplificatore Operazionale} a \textbf{transconduttanza}
\end{itemize}
E vige la seguente equazione:
\begin{equation*}
        Z_x = \frac{E_1}{E_2}R_c
\end{equation*}
In particolare si basa sulla \textbf{tecnica delle proiezioni}:
\begin{center}
    \textbf{Metodo Fast}
\end{center}
\begin{center}
    \includegraphics[width=.4\textwidth]{Images/figure41.png}
\end{center}
\begin{equation*}
    E_0 E_1 cos(\delta_1) +    \cancel{ E_0 E_1 cos(2 \w + \delta_1)}\footnote{Perchè il doppia rampa misura solo la parte continua}
\end{equation*}
Misure che fa lo strumento:
\begin{itemize}
    \item $M_1: \; E_0 E_1 cos(\delta_1)$
    \item $M_2: \; E_0 E_2 cos(\delta_2)$
    \item Sfasiamo di 90 gradi $E_0$
    \item $M_3: \; k E_0 E_1 sin(\delta_1)$
    \item $M_4: \; k E_0 E_2 sin(\delta_2)$
\end{itemize}
Sostituendo nell'equazione di $Z_x$ otteniamo:
\begin{equation*}
    \begin{dcases}
        Z_x = \frac{M_1 + j M_3}{M_2 + j M_4} R_c
    \end{dcases}
\end{equation*}
Da questa equazione possiamo ricavarci\textbf{ L, R, C} etc.\\ \\
\begin{center}
    \textbf{Metodo Medium e Slow}
\end{center}
Con questi metodi, molto \textbf{lenti}, otteniamo una \textbf{precisione maggiore}, grazie a misure con \textbf{sfasamenti diversi}.
\chapter{Diagramma di Dispersione - Velocità di Fase e Velocità di Gruppo}
Consideriamo ora un \textbf{generico modo} con \textbf{pulsazione di taglio} $\omega_t$ e valutiamo per $\omega > \omega_t$ la funzione:
\begin{equation*}
    \omega = \omega(\beta_z)
\end{equation*}
Che ci viene data in forma implicita da:
\begin{equation*}
    \beta^2_z = K^2 - K_t^2
\end{equation*}
Ovvero:
\begin{equation*}
    \beta^2_z = \frac{\omega^2}{c^2} - K_t^2
\end{equation*}
Dividendo ora entrambi i membri per $K^2_t$ otteniamo:
\begin{equation*}
    \frac{\omega^2}{K^2_t c^2}  - \frac{\beta^2_z}{K_t^2} = 1
\end{equation*}
E ricordando che $\omega_t = \frac{K_t}{\sqrt{\epsilon \mu}}$:
\begin{equation*}
    \frac{\omega^2}{\omega^2_t} - \frac{\beta^2_z}{K_t^2} = 1
\end{equation*}
Che non è altro che l'\textbf{equazione di un'iperbole}, i cui \textbf{asintoti} si calcolano facendo tendere $\beta_z \longrightarrow \infty$:
\begin{equation*}
\begin{aligned}
    &\frac{\omega^2}{\omega^2_t} - \frac{\beta^2_z}{K_t^2} = 1 \\
    &\frac{\omega^2}{\omega^2_t} = 1 + \frac{\beta^2_z}{K_t^2}\\
    &\text{Faccio tendere $\beta_z$ all'infinito}\\
    &\frac{\omega^2}{\omega^2_t} = \frac{\beta^2_z}{K_t^2}
\end{aligned}
\end{equation*}
Da cui ci possiamo ricavare l'equazione dell'\textbf{asintoto obliquo}:
\begin{equation*}
    \omega = \frac{\omega_t}{K_t} \beta_z = c \ \beta_z
\end{equation*}
\begin{center}
    \includegraphics[width=\textwidth]{Images/figure45.png}
\end{center}
Consideriamo \textbf{un punto su un ramo di iperbole}, ad esempio quella nera, relativa al \textbf{modo fondamentale} $TE_{10}$, chiamandolo $(\omega_0, \beta_0)$ e tracciamo la \textbf{secante} a questo punto passante per l'origine (la retta rossa), che ha come equazione:
\begin{equation*}
    \omega = \nu_{fase} \beta_z
\end{equation*}
Dove il \textbf{coefficiente angolare} è:
\begin{equation*}
    \nu_{fase} = \frac{\omega_0}{\beta_0}
\end{equation*}
E che dovrebbe rappresentare la \textbf{velocità di propagazione di un segnale di pulsazione} $\omega_0$ sul \textbf{modo fondamentale}.\\ 
Tuttavia come si evince dal disegno $\nu_f > c$ ma questo è \textbf{impossibile}!!\\ \\
Per questo la \textbf{vera velocità} con cui viene trasportata l'informazione non è $\nu_{fase}$ ma:
\begin{equation*}
    \nu_{gruppo} = \frac{d \omega}{d \beta_z}\left(\omega_0 \right)
\end{equation*}
Che non è altro che il \textbf{coefficiente angolare della retta tangente al punto} (linea verde) che è \textbf{più piccolo della velocità della luce}.\\ \\
Ritorniamo ora a questa relazione:
\begin{equation*}
    \frac{\omega^2}{\omega^2_t} = 1 + \frac{K^2_z}{K_t^2}
\end{equation*}
\begin{equation*}
    \omega^2 = \omega^2_t + \omega^2_t \frac{K^2_z}{K_t^2}
\end{equation*}
\begin{equation*}
    \omega = \sqrt{\omega^2_t + c^2 K^2_z}
\end{equation*}
Ed ora differenziamola:
\begin{equation*}
    \frac{d \omega}{d K_z} = \frac{1}{2} \frac{1}{\underbrace{\sqrt{\omega^2_t + c^2 K^2_z}}_{\omega}} 2 c^2 K_z = \frac{c^2 K_z}{\omega} = \nu_{gruppo} = c^2 \cdot \frac{1}{\nu_f}
\end{equation*}
\end{document}
