\chapter{Single Link}

\section{Parametri di Canale}
\subsection{Capacità di canale}
La \textbf{Capacità} di un canale è il massimo rate a cui è possibile trasmettere, avendo una probabilità di errore piccola a piacere:
\begin{equation*}
    C = B\log_2(1 + SNR) \ [bit/s]
\end{equation*}

Da notare è che la capacità non dipende nè dalla modulazione nè dalla codifica di canale in uso.
\\

\subsection{Efficienza Energetica}
L'\textbf{Efficienza Energetica} di un canale di comunicazione misura quanti bit possono essere trasmessi in maniera affidabile per Joule di energia consumata:
\begin{equation*}
    EE =\frac{C}{P_t} = \frac{B\log_2(1 + SNR)}{\mu p + P_c}  \ [bit/J]
\end{equation*}

dove:
\begin{itemize}
    \item $\mu \geq 1$ è l'inverso dell'efficienza dell'amplificazione di trasmissione
    \item $P_c$ è la potenza generata dai circuiti del sistema che dipende dal sistema utilizzato.
\end{itemize}
%------------------------------------------------------------------------------------%
\pagebreak 


\section{Canale SISO (Single Input Single Output)}
Consideriamo una singola antenna in trasmissione e in ricezione.\\

Il segnale ricevuto è:
\begin{equation*}
    r = \sqrt{p} h s + n
\end{equation*}

L'SNR sarà: 
\begin{equation*}
    SNR = \frac{|\sqrt{p}h|^2 \av{|s|^2}} {\av{|n|^2}} = \frac{p|h|^2}{\sigma^2}
\end{equation*}

La \textbf{Capacità} del canale \textbf{SISO} è:
\begin{equation*}
    C_{SISO} = B\log_2 \left(1 +  \frac{p|h|^2}{\sigma^2} \right) \ [bit/s]
\end{equation*}

L'\textbf{Efficienza Energetica} del canale \textbf{SISO} è:
\begin{equation*}
    EE_{SISO} =\frac{B\log_2 \left(1 +  \frac{p|h|^2}{\sigma^2} \right)}{\mu p + P_c} \ [bit/J]
\end{equation*}
%------------------------------------------------------------------------------------%
\pagebreak 


\section{Canale SIMO (Single Input Multiple Output)}
Consideriamo un canale \textbf{Flat-Fading} per una trasmissione in cui abbiamo una sola antenna trasmittente (antenna che immette l'input) e più antenne riceventi (antenne di output).\\

Il segnale ricevuto dall'l-esima antenna è:
\begin{equation*}
    r_l = \sqrt{p} h_l s + n
\end{equation*}

Quindi la BS riceve un vettore L x 1 dato da:
\begin{equation*}
    \vc{r} = \begin{bmatrix}
    r_1\\.\\.\\.\\r_L
    \end{bmatrix} = 
    \sqrt{p} \begin{bmatrix}
    h_1\\.\\.\\.\\h_L
    \end{bmatrix} s + 
    \begin{bmatrix}
    n_1\\.\\.\\.\\n_L
    \end{bmatrix} = \sqrt{p} \ \vc{h} \ s + \vc{n}
\end{equation*}

$\vc{n} \sim  CN(0,\sigma^2 \vc{I})$ i.i.d.
\\

\subsection{Capacità del canale SIMO}
\begin{equation*}
    C_{SIMO} = B\log_2 \left(1 + \frac{p||\vc{h}||^2}{\sigma^2}\right)  \ [bit/s]
\end{equation*}
\\

\subsection{EE del canale SIMO}
\begin{equation*}
    E_{SIMO} = \frac{B\log_2 \left(1 + \frac{p||\vc{h}||^2}{\sigma^2}\right)}{\mu p + P_c}  \ [bit/J]
\end{equation*}
\\

\subsection{Filtro Adattato / MRC per Canali SIMO}
Si può raggiungere la capacità del canale SIMO \textbf{filtrando linearmente} il vettore ricevuto:
\begin{equation*}
    y = \vc{c}^H \vc{r} = \sqrt{p} \ \vc{c}^H \vc{h}\ s + \vc{c}^H \vc{n} = \sqrt{p}z+w
\end{equation*}

Quindi dopo il filtro l'SNR sarà:
\begin{equation*}
    SNR = \frac{p\av{|z|^2}}{\av{|w|^2}}
\end{equation*}

Esaminando il termine al numeratore:
\begin{equation*}
   p\av{|z|^2} = p\av{|s|^2} \ |\vc{c}^H \vc{h}|^2 = p|\vc{c}^H \vc{h}|^2 \leq p \ ||\vc{c}||^2 \ ||\vc{h}||^2 \ \text{Per Shwartz}
\end{equation*}

Quindi si raggiunge l'uguaglianza quando \vc{c} e \vc{h} sono \textbf{paralleli}.\\

Esaminando ora il denominatore, capiamo che w è ancora una variabile aleatoria Gaussiana, quindi:

\begin{equation*}
\begin{aligned}
    &\av{w} = \vc{c}^H \ \av{n} = 0 \\
    &\av{|w|^2} = \av{w \ w^*} = \vc{c}^H \av{\vc{n} \ \vc{n}^H} \ \vc{c} = \sigma^2 ||\vc{c}||^2
\end{aligned}
\end{equation*}

Quindi Il \textbf{Massimo SNR} ottenibile dopo il filtro è:
\begin{equation*}
    SNR =\frac{p|\vc{h}|^2 \cancel{||\vc{c}||^2} }{\sigma^2 \cancel{||\vc{c}||^2}} = \frac{p||\vc{h}||^2}{\sigma^2}
\end{equation*}

che può essere ottenuto scegliendo \vc{c} = $\alpha$ \vc{h}.\\

Tale filtro si dice \textbf{Filtro Adattato} oppure \textbf{Maximum Reciver Combining} (MRC)
\\

\subsection{Antenna Selection}
Con il metodo di \textbf{Antenna Selection} si riceve solo dall'antenna con il canale migliore:
\begin{equation*}
    \hat{h} = \max_{l\in[1,L]} |h_l|
\end{equation*}

Il suo SNR sarà:
\begin{equation*}
    SNR = \frac{p|\hat{h}|^2}{\sigma^2}
\end{equation*}

Quindi avremmo:
\begin{equation*}
    \begin{aligned}
    &R = \rate{p|\hat{h}|^2}{\sigma^2}
    \\[10pt]
    &EE = \frac{\rate{p|\hat{h}|^2}{\sigma^2}}{\mu p + P_c}
    \end{aligned}
\end{equation*}
%------------------------------------------------------------------------------------%
\pagebreak 


\section{Canale MISO (Multiple Input Single Output)}
Consideriamo un canale \textbf{Flat-Fading} per una trasmissione in cui abbiamo una sola antenna ricevente (antenna di output) e più antenne trasmittenti (antenne che immettono l'input).\\

Il canale che va dall'l-esima antenna della BS verso ciascun terminale mobile è:
\begin{equation*}
    y_l = \sqrt{p} \ q_l \ g_l^* \ s
\end{equation*}

dove:
\begin{itemize}
    \item $g_l^*$ è il Coefficiente di Downlink Utente-L-esima antenna trasmittente
    \item $q_l$ è il Coefficiente di Beamforming dell'l-esima antenna trasmittente, tale che $||\vc{q}||^2 = 1$
    \item La potenza del segnale trasmesso sarà quindi:  $P = \sum_{l=1}^{L} p |q_l|^2 = p ||\vc{q}||^2 = p$
\end{itemize}

Quindi il terminale mobile riceve la sovrapposizione dei segnale dalle L antenne della BS:
\begin{equation*}
    r = \sqrt{p} \left(\sum_{l=1}^{L} g_l^* \ q_l \right) \ s + n = \sqrt{p} \ \vc{g}^H \ \vc{q} \ s + n
\end{equation*}


\subsection{MTC (Maximum Transmit Combining)}
Dato il segnale ricevuto:
\begin{equation*}
    r = \sqrt{p} \ \vc{g}^H \ \vc{q} \ s + n = \sqrt{p}\ z + n
\end{equation*}

L'SNR al ricevitore sarà:
\begin{equation*}
    SNR = \frac{p\av{|z|^2}}{\av{|n|^2}}
\end{equation*}

Esaminando il numeratore:
\begin{equation*}
   p\av{|z|^2} = p\av{|s|^2} \ |\vc{g}^H \ \vc{q}|^2 \leq p ||\vc{g}||^2 \ ||\vc{q}||^2  = p ||\vc{g}||^2
\end{equation*}
dove l'uguaglianza vale quando \vc{g} e \vc{q} sono paralleli.

Quindi il massimo SNR ottenibile dopo il filtro sarà:
\begin{equation*}
    SNR = \frac{p ||\vc{g}||^2}{\sigma^2} \implies C_{MISO} = \rate{p ||\vc{g}||^2}{\sigma^2}
\end{equation*}

e può essere ottenuto scegliendo $\vc{q} = \alpha \vc{g}$. \\

Di solito si prende $\alpha = \frac{1}{||\vc{g}||}$ in modo tale da normalizzare $\vc{q}$.
\\

\subsection{Antenna Selection}
Con il metodo di \textbf{Antenna Selection} si trasmette solo dall'antenna con il canale migliore:
\begin{equation*}
    \hat{g} = \max_{l\in[1,L]} |g_l|
\end{equation*}

Il suo SNR sarà:
\begin{equation*}
    SNR = \frac{p|\hat{g}|^2}{\sigma^2}
\end{equation*}

Quindi avremmo:
\begin{equation*}
    \begin{aligned}
    &R = \rate{p|\hat{g}|^2}{\sigma^2}
    \\[10pt]
    &EE = \frac{\rate{p|\hat{g}|^2}{\sigma^2}}{\mu p + P_c}
    \end{aligned}
\end{equation*}
%------------------------------------------------------------------------------------%
\pagebreak 


\section{Canale MIMO (Multiple Input Multiple Output)}
Consideriamo un canale \textbf{Flat-Fading} per una trasmissione in cui abbiamo $N_T$ antenne al $T_x$ e $N_R$ antenne al $R_x$ e trasmettiamo $N_T$ simboli contemporaneamente $s_1, ..., s_{N_T}$.\\
\\

\subsection{Trasmissione}
Il Segnale $x_{i}$ trasmesso dall'antenna i-esima è una sovrapposizione degli $N_T$ simboli:
\begin{equation*}
    \begin{aligned}
    x_1 &= \sqrt{p} \sum_{nt=1}^{N_T} s_{nt} \ q_{nt,1} \\
    x_2 &= \sqrt{p} \sum_{nt=1}^{N_T} s_{nt} \ q_{nt,2} \\
    &.\\&.\\&.\\
    x_{N_T} &= \sqrt{p} \sum_{nt=1}^{N_T} s_{nt} \ q_{nt,N_T} \\
    \end{aligned}
\end{equation*}
con $q_{nt,i}$ il coefficiente di Beamforming dell'antenna i, relativo al simbolo nt.

Ora definiamo:
\begin{itemize}
    \item $\vc{x} = [x_1, ..., x_{N_T}]$
    \item $\vc{M} = \bigl(
    \begin{smallmatrix}
    q_{1,1} & q_{2,1} & \dots &  q_{N_T,1} \\
    q_{1,2} & \dots & \\
    \vdots\\
    q_{1,N_T} & \dots & \dots & q_{N_T,N_T}
    \end{smallmatrix}
    \bigr) $
\end{itemize}
Per ottenere:
\begin{equation*}
    \vc{x} = \sqrt{p} \vc{M} \ \vc{s}
\end{equation*}

Dove vale il seguente vincolo:
\begin{equation*}
    tr(\vc{M}\vc{M}^H) = 1
\end{equation*}

Dato che se assumiamo $\av{\vc{s}\vc{s}^H} = \vc{I}$:
\begin{equation*}
    P_t = \av{||\vc{x}||^2} = \av{tr(\vc{x}\vc{x}^H)} = p \ tr(\vc{M}\vc{M}^H)
\end{equation*}

La potenza dello stream nt è:
\begin{equation*}
    P_{nt} = \av{|\sqrt{p}\vc{q}_{nt}^T \vc{s}|^2} = p||\vc{q}_{nt}||^2 \av{\vc{s}\vc{s}^H} = p||\vc{q}_{nt}||^2
\end{equation*}
Allora la \textbf{Potenza Totale} può essere calcolata così:
\begin{equation*}
    \sum_{nt=1}^{N_T} P_{nt} = \sum_{nt=1}^{N_T} p||\vc{q_{nt}}||^2 = p tr(\vc{M}\ \vc{M}^H) = p
\end{equation*}
\\

\subsection{Ricezione}
Ad ogni antenna ricevente arriva la sovrapposizione dei segnali trasmessi da tutte le antenne trasmittenti, quindi il segnale ricevuto dall'antenna nr sarà:
\begin{equation*}
    r_{nr} = \sum_{nt=1}^{N_T} h_{nr,nt} \ x_{nt} + n_{nr}
\end{equation*}

con $h_{nr,nt}$ canale tra l'antenna nr ricevente e l'antenna nt trasmittente.

\begin{equation*}
    \begin{aligned}
    r_1 &=  \sum_{nt=1}^{N_T} h_{1,nt} x_{nt} + n_{1} \\
    r_2 &=  \sum_{nt=1}^{N_T} h_{2,nt} x_{nt} + n_{2} \\
    &.\\&.\\&.\\
    r_{N_R} &=  \sum_{nt=1}^{N_T} h_{N_R} x_{nt} + n_{N_R} \\
    \end{aligned}
\end{equation*}
\\

Che possiamo sintetizzare così:
\begin{equation*}
    \begin{aligned}
     r_1 &=   \vc{h}^T_{1} \vc{x} + n_{1} \\
    r_2 &=  \vc{h}^T_{2} \vc{x} + n_{2} \\
    &.\\&.\\&.\\
    r_{N_R} &=  \vc{h}^T_{N_R} \vc{x} + n_{N_R} \\
    \end{aligned}
\end{equation*}
\\

Infine definendo la matrice:
\begin{equation*}
    \vc{H}=
    \begin{pmatrix}
    h_{1,1} & \dots & h_{1,N_T} \\
    \vdots & \vdots & \vdots \\
    h_{N_R,1} & \dots & h_{N_R,N_T}
    \end{pmatrix}
\end{equation*}

Otterremo il seguente modello:
\begin{equation*}
    \vc{r} = \vc{H} \ \vc{x} + \vc{n} = \sqrt{p} \ \vc{H} \ \vc{M} \ \vc{s} + \vc{n}
\end{equation*}
\\

\subsection{Capacità del canale MIMO}
La Capacità del canale MIMO sarà:
\begin{equation*}
    C_{MIMO} = B\log_2 \left(\frac{|\av{\vc{r} \ \vc{r}^H}|}{|\av{\vc{n} \ \vc{n}^H}|}\right) \ [bit/s]
\end{equation*}

Assumiamo \vc{x} e \vc{n} indipendenti, allora:\\
\begin{equation*}
    \begin{aligned}
    \av{\vc{r} \ \vc{r}^H} & = \av{(\vc{H} \ \vc{x} + \vc{n}) \ (\vc{H} \ \vc{x} + \vc{n})^H} = \\ \\
    & = \vc{H} \ \underbrace{\av{\vc{x} \ \vc{x}^H}}_{p\vc{M} \ \vc{M}^H = p \ \vc{Q}} \vc{H}^H + \underbrace{\av{\vc{n} \ \vc{n}^H}}_{\sigma^2\vc{I}} = \\ \\
    & = p\vc{H}\vc{Q}\vc{H}^H + \sigma^2\vc{I}
    \end{aligned}
\end{equation*}
\\

Quindi la Capacità nel canale MIMO si scriverà:
\begin{equation*}
    C_{MIMO} = B\log_2 \left(\frac{| p\vc{H}\vc{Q}\vc{H}^H + \sigma^2\vc{I}|}{|\sigma^2\vc{I}|}\right) = B\log_2 \left(\left| \vc{I} + \frac{p}{\sigma^2}\vc{H}\vc{Q}\vc{H}^H \right|\right)  \ [bit/s]
\end{equation*}
\\

\subsubsection{Disuguaglianza di Hadamard}
Sia $\vc{A} \in \mathbb{R}^{n X n}$ definita positiva ($\Re (\vc{x}^* \vc{A}\vc{x})>0 \; \forall \vc{x} \neq \vc{0} \in \mathbb{C}^n$), allora:
\begin{equation*}
    |\vc{A}| = \leq \prod_{i=1}^n A_{i,i}
\end{equation*}

Se $\vc{A}$ è diagonale $\implies A_{i,i} = \lambda_i$
\\

\subsubsection{Applicazione}
La matrice $\vc{I} + \frac{p}{\sigma^2}\vc{H}\vc{Q} \vc{H}^{H}$ è definita positiva:
\begin{empheq}[box=\tcbhighmath]{equation*}
\begin{aligned}
Dimostrazione \\
\vc{x}^H \left(\vc{I} + \frac{p}{\sigma^2}\vc{H}\vc{Q} \vc{H}^{H} \right) \ \vc{x} &= ||\vc{x}||^2 + \frac{p}{\sigma^2} \vc{x}^H \vc{H}\vc{M} \ \vc{M}^H \vc{H}^H \vc{x} \\
&= ||\vc{x}||^2 + \frac{p}{\sigma^2} ||\vc{M}^H \vc{H} \vc{x}||^2 > 0
\end{aligned}
\end{empheq}
\\

Quindi l'autovalore i-esimo della matrice $\left(\vc{I} + \frac{p}{\sigma^{2}} \vc{H} \vc{Q} \vc{H}^{H} \right)$ sarà $ \implies  1 + \frac{p}{\sigma^2} \ \lambda_i$, con $\lambda_i$ che è l'autovalore della matrice $\vc{H}\vc{Q} \vc{H}^{H}$. \\

In aggiunta ogni autovettore  \vc{v} di  $\vc{H}\vc{Q} \vc{H}^{H}$ sarà anche autovettore di $\vc{I} + \frac{P}{\sigma^2}\vc{H}\vc{Q} \vc{H}^{H}$ dato che:
\begin{empheq}[box=\tcbhighmath]{equation*}
\begin{aligned}
Dimostrazione \\
\left(\vc{I} + \frac{p}{\sigma^2}\vc{H}\vc{Q} \vc{H}^{H} \right) \ \vc{v} &= \vc{v} + \frac{p}{\sigma^2}\vc{H}\vc{Q} \vc{H}^{H} \vc{v} =\\
&= \vc{v} + \frac{p}{\sigma^2} \ \lambda_i \vc{v} = \\
&= (1 + \frac{p}{\sigma^2} \ \lambda_i) \vc{v}
\end{aligned}
\end{empheq}
\\

Allora applicando la disuguaglianza di Hadamard si ha:
\begin{equation*}
    C \leq B \sum_{i=1}^{min\{N_R, N_T\}} \log_2 \left(1 + \frac{p}{\sigma^2} \lambda_i \right)
\end{equation*}

\begin{center}
    Ma come posso rendere $\vc{I} + \frac{p}{\sigma^{2}} \vc{H} \vc{Q} \vc{H}^{H}$ anche \textbf{DIAGONALE}????
\end{center}

\begin{itemize}
    \item Scriviamo la matrice di canale attraverso la sua SVD: $\vc{H} = \vc{U} \vc{\Lambda} \vc{V}^H$
    \item Scriviamo la matrice di covarianza del segnale tramite la sua EVD: $\vc{Q} = \vc{U}_Q \vc{\Lambda}_Q \vc{U}_Q^H$
\end{itemize}

Così sostituendo otterremo:
\begin{equation*}
    \vc{H} \vc{Q} \vc{H}^{H} = \underbrace{\vc{U} \vc{\Lambda} \vc{V}^H}_{\vc{H}} \ \underbrace{\vc{U}_Q \vc{\Lambda}_Q \vc{U}_Q^H}_{Q} \ \underbrace{\vc{V} \vc{\Lambda}^H \vc{U}^H}_{\vc{H}^H}
\end{equation*}

Ma se scegliamo $\vc{U}_Q = \vc{V}$ otteniamo:
\begin{equation*}
    \vc{I} + \frac{p}{\sigma^{2}} \vc{U} \vc{\Lambda} \vc{\Lambda} _Q \vc{\Lambda}^H \vc{U}^H
\end{equation*}
Che ci fa arrivare alla massima capacità.\\

\begin{center}
    \textbf{Ma quindi come dobbiamo elaborare il segnale al trasmettitore e al ricevitore per raggiungere la massima capacità???}  
\end{center}

\begin{equation*}
\begin{aligned}
     \vc{r} &= \vc{H} \vc{x} + \vc{n} = \sqrt{p} \vc{H} \vc{M} \vc{s} + \vc{n} = \\
     &= \sqrt{p} \  \underbrace{\vc{U} \vc{\Lambda} \vc{V}^H}_{\vc{H}} \ \underbrace{\vc{U}_M \vc{\Lambda}_M \vc{V}_M^H}_{\vc{M}} \ \vc{s} + \vc{n} = \\
     &= \sqrt{p} \ \vc{U} \vc{\Lambda} \vc{V}^H \ \vc{U}_Q \vc{\Lambda}_Q^{1/2} \tilde{\vc{s}} + \vc{n}
\end{aligned}
\end{equation*}

dove:
\begin{itemize}
    \item $\vc{U}_M = \vc{U}_Q$ e $\vc{\Lambda}_M = \vc{\Lambda}_Q^{1/2}$
    \item $\tilde{\vc{s}} = \vc{V}_{M}^H \vc{s}$
\end{itemize}

Scegliendo $\vc{U}_Q = \vc{V}$ otteniamo:
\begin{equation*}
    \vc{r} = \sqrt{p} \ \vc{U} \vc{\Lambda} \vc{\Lambda}_Q^{1/2} \tilde{\vc{s}} + \vc{n}
\end{equation*}

Al ricevitore filtriamo il segnale ricevuto con $\vc{C}^H = \vc{U}^H$:\\

\begin{equation*}
\begin{aligned}
     \vc{y} &= \vc{C}^H \vc{r} = \sqrt{p} \vc{C}^H \vc{U} \vc{\Lambda} \vc{\Lambda}_Q^{1/2} \tilde{\vc{s}} + \vc{C}^H \vc{n} = \\
     &= \sqrt{p} \vc{\Lambda} \vc{\Lambda}_Q^{1/2} \tilde{\vc{s}} + \vc{U}^H \vc{n} = \sqrt{p} \vc{\Lambda} \vc{\Lambda}_Q^{1/2} \tilde{\vc{s}} + \vc{w}
\end{aligned}
\end{equation*}

Scriviamolo componente per componente:\\
\begin{equation*}
    \vc{y} = \begin{pmatrix}
    y_1 \\ y_2 \\ ... \\ y_{N_R}
    \end{pmatrix} = \sqrt{p} \begin{pmatrix}
    \lambda_{1,H} \sqrt{\lambda_{Q,1}} \ \tilde{s}_1 + w_1 \\
    \lambda_{2,H} \sqrt{\lambda_{Q,2}} \ \tilde{s}_2 + w_2 \\
    .\\.\\.\\
    \lambda_{N_R,H} \sqrt{\lambda_{Q,N_R}} \ \tilde{s}_1 + w_{N_R}
    \end{pmatrix}
\end{equation*}

Così facendo abbiamo ottenuto $min\{N_R, N_T\}$ Canali SISO Paralleli.\\

Quindi la capacità del Canale MIMO può anche essere calcolata sommando le capacità dei singoli canali SISO.\\

Il Canale SISO i-esimo ha il seguente SNR:
\begin{equation*}
    SNR_i = \frac{p}{\sigma^2} \lambda_{i,H}^2 \lambda_{i,Q}
\end{equation*}

Quindi la Capacità del Canale SISO i-esimo sarà:
\begin{equation*}
    C_i = B\log_2\left(1 + \frac{p}{\sigma^2} \lambda_{i,H}^2 \lambda_{i,Q} \right)
\end{equation*}
